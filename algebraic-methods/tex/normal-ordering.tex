\documentclass[11pt,fleqn]{article}
\usepackage[cm]{fullpage}
\usepackage{mathtools} %includes amsmath
\usepackage{amsfonts}
\usepackage{bm}
\usepackage{xfrac}
\usepackage{url}
%greek letters
\renewcommand{\a}{\alpha}    %alpha
\renewcommand{\b}{\beta}     %beta
\newcommand{\g}{\gamma}      %gamma
\newcommand{\G}{\Gamma}      %Gamma
\renewcommand{\d}{\delta}    %delta
\newcommand{\D}{\Delta}      %Delta
\newcommand{\e}{\varepsilon} %epsilon
\newcommand{\ev}{\epsilon}   %epsilon*
\newcommand{\z}{\zeta}       %zeta
\newcommand{\h}{\eta}        %eta
\renewcommand{\th}{\theta}   %theta
\newcommand{\Th}{\Theta}     %Theta
\newcommand{\io}{\iota}      %iota
\renewcommand{\k}{\kappa}    %kappa
\newcommand{\la}{\lambda}    %lambda
\newcommand{\La}{\Lambda}    %Lambda
\newcommand{\m}{\mu}         %mu
\newcommand{\n}{\nu}         %nu %xi %Xi %pi %Pi
\newcommand{\p}{\rho}        %rho
\newcommand{\si}{\sigma}     %sigma
\newcommand{\siv}{\varsigma} %sigma*
\newcommand{\Si}{\Sigma}     %Sigma
\renewcommand{\t}{\tau}      %tau
\newcommand{\up}{\upsilon}   %upsilon
\newcommand{\f}{\phi}        %phi
\newcommand{\F}{\Phi}        %Phi
\newcommand{\x}{\chi}        %chi
\newcommand{\y}{\psi}        %psi
\newcommand{\Y}{\Psi}        %Psi
\newcommand{\w}{\omega}      %omega
\newcommand{\W}{\Omega}      %Omega
%ornaments
\newcommand{\eth}{\ensuremath{^\text{th}}}
\newcommand{\rst}{\ensuremath{^\text{st}}}
\newcommand{\ond}{\ensuremath{^\text{nd}}}
\newcommand{\ord}[1]{\ensuremath{^{(#1)}}}
\newcommand{\dg}{\ensuremath{^\dagger}}
\newcommand{\bigo}{\ensuremath{\mathcal{O}}}
\newcommand{\tl}{\ensuremath{\tilde}}
\newcommand{\ol}[1]{\ensuremath{\overline{#1}}}
\newcommand{\ul}[1]{\ensuremath{\underline{#1}}}
\newcommand{\op}[1]{\ensuremath{\hat{#1}}}
\newcommand{\ot}{\ensuremath{\otimes}}
\newcommand{\wg}{\ensuremath{\wedge}}
%text
\newcommand{\tr}{\ensuremath{\hspace{1pt}\mathrm{tr}\hspace{1pt}}}
\newcommand{\Alt}{\ensuremath{\mathrm{Alt}}}
\newcommand{\sgn}{\ensuremath{\mathrm{sgn}}}
\newcommand{\occ}{\ensuremath{\mathrm{occ}}}
\newcommand{\vir}{\ensuremath{\mathrm{vir}}}
\newcommand{\spn}{\ensuremath{\mathrm{span}}}
\newcommand{\vac}{\ensuremath{\mathrm{vac}}}
\newcommand{\bs}{\ensuremath{\text{\textbackslash}}}
\newcommand{\im}{\ensuremath{\mathrm{im}\hspace{1pt}}}
\renewcommand{\sp}{\hspace{30pt}}
%dots
\newcommand{\ld}{\ensuremath{\ldots}}
\newcommand{\cd}{\ensuremath{\cdots}}
\newcommand{\vd}{\ensuremath{\vdots}}
\newcommand{\dd}{\ensuremath{\ddots}}
\newcommand{\etc}{\ensuremath{\mathinner{\mkern-1mu\cdotp\mkern-2mu\cdotp\mkern-2mu\cdotp\mkern-1mu}}}
%fonts
\newcommand{\bmit}[1]{{\bfseries\itshape\mathversion{bold}#1}}
\newcommand{\mc}[1]{\ensuremath{\mathcal{#1}}}
\newcommand{\mb}[1]{\ensuremath{\mathbb{#1}}}
\newcommand{\mf}[1]{\ensuremath{\mathfrak{#1}}}
\newcommand{\mr}[1]{\ensuremath{\mathrm{#1}}}
\newcommand{\bo}[1]{\ensuremath{\mathbf{#1}}}
%styles
\newcommand{\ts}{\textstyle}
\newcommand{\ds}{\displaystyle}
\newcommand{\phsub}{\ensuremath{_{\phantom{p}}}}
\newcommand{\phsup}{\ensuremath{^{\phantom{p}}}}
%fractions, derivatives, parentheses, brackets, etc.
\newcommand{\pr}[1]{\ensuremath{\left(#1\right)}}
\newcommand{\brk}[1]{\ensuremath{\left[#1\right]}}
\newcommand{\fr}[2]{\ensuremath{\dfrac{#1}{#2}}}
\newcommand{\pd}[2]{\ensuremath{\frac{\partial#1}{\partial#2}}}
\newcommand{\pt}{\ensuremath{\partial}}
\newcommand{\br}[1]{\ensuremath{\langle#1|}}
\newcommand{\kt}[1]{\ensuremath{|#1\rangle}}
\newcommand{\ip}[1]{\ensuremath{\langle#1\rangle}}
\newcommand{\NO}[1]{\ensuremath{{\bm{:}}#1{}{\bm{:}}}}
\newcommand{\floor}[1]{\ensuremath{\left\lfloor#1\right\rfloor}}
\newcommand{\ceil}[1]{\ensuremath{\left\lceil#1\right\rceil}}
\usepackage{stackengine}
\newcommand{\GNO}[1]{\setstackgap{S}{0.7pt}\ensuremath{\Shortstack{\textbf{.} \textbf{.} \textbf{.}}#1\Shortstack{\textbf{.} \textbf{.} \textbf{.}}}}
\newcommand{\cmtr}[2]{\ensuremath{[\cdot,#2]^{#1}}}
\newcommand{\cmtl}[2]{\ensuremath{[#2,\cdot]^{#1}}}
%structures
\newcommand{\eqn}[1]{(\ref{#1})}
\newcommand{\ma}[1]{\ensuremath{\begin{bmatrix}#1\end{bmatrix}}}
\newcommand{\ar}[1]{\ensuremath{\begin{matrix}#1\end{matrix}}}
\newcommand{\miniar}[1]{\ensuremath{\begin{smallmatrix}#1\end{smallmatrix}}}
%contractions
\usepackage{simplewick}
\usepackage[nomessages]{fp}
\newcommand{\ctr}[6][0]{\FPeval\height{0.6+#1*0.5}\ensuremath{\contraction[\height ex]{#2}{#3}{#4}{#5}}}
\newcommand{\ccr}[4]{\ctr[0.7]{#1}{#2}{#3}{#4}{}\ctr{#1}{#2}{#3}{#4}}
\usepackage{calc}
\makeatletter
\def\@hspace#1{\begingroup\setlength\dimen@{#1}\hskip\dimen@\endgroup}
\makeatother
\newcommand{\halfphantom}[1]{\hspace{\widthof{#1}*\real{0.5}}}
\newcommand{\fullctr}[1]{\ensuremath{\contraction[0.5ex]{}{\vphantom{#1}}{\hphantom{#1}}{}{}{}\contraction[0.5ex]{}{\vphantom{#1}}{\halfphantom{#1}}{}{}{}#1}}
%math sections
\usepackage{cleveref}
\usepackage{amsthm}
\usepackage{thmtools}
\declaretheoremstyle[spaceabove=10pt,spacebelow=10pt,bodyfont=\small]{mystyle}
\theoremstyle{mystyle}
\newtheorem{dfn}{Definition}[section]
\crefname{dfn}{definition}{definitions}
\Crefname{dfn}{Def}{Defs}
\newtheorem{thm}{Theorem}[section]
\crefname{thm}{theorem}{theorems}
\Crefname{thm}{Thm}{Thms}
\newtheorem{cor}{Corollary}[section]
\crefname{cor}{corollary}{corollaries}
\Crefname{cor}{Cor}{Cors}
\newtheorem{lem}{Lemma}[section]
\crefname{lem}{lemma}{lemmas}
\Crefname{lem}{Lem}{Lems}
\newtheorem{rmk}{Remark}[section]
\crefname{rmk}{remark}{remarks}
\Crefname{rmk}{Rmk}{Rmks}
\newtheorem{pro}{Proposition}[section]
\crefname{pro}{proposition}{propositions}
\Crefname{pro}{Prop}{Props}
\newtheorem{ntt}{Notation}[section]
\crefname{ntt}{notation}{notations}
\Crefname{ntt}{Notation}{Notations}
\newcommand{\logbox}{\ensuremath{\text{\makebox[\widthof{exp}][c]{ln}}}}
\newcommand{\expbox}{\ensuremath{\text{\makebox[\widthof{exp}][c]{exp}}}}

\numberwithin{equation}{section}
\usepackage{cancel}
\usepackage{scrextend}

\newcommand{\hole}{\circ}
\newcommand{\ptcl}{\bullet}
\newcommand{\circled}[1]{\raisebox{.5pt}{\textcircled{\raisebox{-.9pt}{#1}}}}



\author{Andreas V. Copan}
\date{}
\title{Fock space algebraic methods:\\Normal ordering with respect to $\vac$ and $\F$}

%%%DOCUMENT%%%
\begin{document}

\maketitle
\tableofcontents

\newpage
\section{Fock space}

\begin{dfn}
\label{fock-space}
\bmit{Fock space.}
Let $\mc{H}$ be a complete one-particle Hilbert space, so that $\mc{H}^{\otimes n}=\underset{\text{$n$ times}}{\mc{H}\otimes\cd\otimes\mc{H}}$ is an $n$-particle Hilbert space.
The space of candidate $n$-fermion wavefunctions in $\mc{H}^{\otimes n}$ is then $\mb{A}(\mc{H}^{\otimes n})$, its antisymmetric subspace.
If $\{\y_p\}$ is a complete, orthonormal basis for $\mc{H}$, then
\begin{align}
\label{fock-basis}
&&\ts
  \F_{(p_1\cd p_n)}
=
  \fr{1}{\sqrt{n!}}
  \sum_\pi^{\mr{S}_n}
  \e_{\pi}
  \y_{p_{\pi(1)}}\otimes\cd\otimes\y_{p_{\pi(n)}}
\sp
  \text{where }
  p_1<\cd<p_n
\end{align}
is a complete, orthonormal basis for $\mb{A}(\mc{H}^{\otimes n})$.
Represented as functions of space and spin coordinates, these basis vectors become Slater determinants.
The \textit{fermionic Fock space} $F(\mc{H})$ over $\mc{H}$ is given by a direct sum of $\mb{A}(\mc{H}^{\otimes n})$ from $n=0$ to $n=\infty$, i.e. $F(\mc{H})=\mb{C}\oplus \mb{A}(\mc{H}^{\otimes 1})\oplus\mb{A}(\mc{H}^{\otimes 2})\oplus\mb{A}(\mc{H}^{\otimes 3})\oplus\cd$, which spans all possible fermionic wavefunctions.
\end{dfn}

\begin{dfn}
\label{on-representation}
\bmit{The occupation number representation of $F(\mc{H})$.}
In the \textit{occupation number representation} of Fock space, the basis vectors (equation \ref{fock-basis}) are represented as lists of 1s and 0s
\begin{align}
\label{occupation-vector}
&&
  \kt{(p_1\cd p_n)}
\equiv
  \kt{n_1,n_2,n_3,\ld,n_{\infty}}
\sp\text{where }
  p_1<\cd<p_n
\text{ and }
  n_p
=
  \left\{\ar{
    1 & \text{if $p\in(p_1\cd p_n)$}\\
    0 & \text{if $p\notin(p_1\cd p_n)$}
  }\right.
\end{align}
where $n_p=1$ when $\y_p$ is occupied and $n_p=0$ when $\y_p$ is unoccupied.
The basis for $\mb{A}(\mc{H}^{\otimes n})$ is given by distributing $n$ 1s in the occupation vector in all possible ways.
The state in which no spin-orbitals are occupied is called the \textit{vacuum state}, here denoted $\kt{\vac}$, which spans $\mc{H}^{\otimes0}$ ($=\mb{A}(\mc{H}^{\otimes0})=\mb{C}$, up to isomorphism).
\end{dfn}


\section{Particle-hole operators}

\begin{dfn}
\bmit{Particle-hole operators.}
\textit{Particle-hole operators} include \textit{annihilation operators} and \textit{creation operators}.\\
The \textit{annihilation operator} of $\y_p$ is a linear mapping $a_p:\mb{A}(\mc{H}^{\otimes n})\rightarrow\mb{A}(\mc{H}^{\otimes (n-1)})$ defined by
\begin{align}
&&
\label{annihilation-operator}
  a_p
  \kt{\cd n_p\cd}
=
  (-)^{n_1+\cd+n_{p-1}}
  \kt{\cd n_p-1\cd}
\ \ \text{ if $n_p=1$}
&&
  a_p\kt{\cd n_p\cd}
=
  0
\ \ \text{ if $n_p=0$}
\end{align}
and the \textit{creation operator} of $\y_p$ is a linear mapping $c_p:\mb{A}(\mc{H}^{\otimes n})\rightarrow\mb{A}(\mc{H}^{\otimes(n+1)})$ defined by
\begin{align}
\label{creation-operator}
&&
  c_p
  \kt{\cd n_p\cd}
=
  (-)^{n_1+\cd+n_{p-1}}
  \kt{\cd n_p+1\cd}
\ \ \text{ if $n_p=0$}
&&
  c_p\kt{\cd n_p\cd}
=
  0
\ \ \text{ if $n_p=1$.}
\end{align}
\end{dfn}

\begin{pro}
\label{particle-hole-adjoint-relation}
\bmit{$c_p=a_p\dg$.}
\textit{Creation and annihilation operators of a one-particle state $\y_p$ are adjoints of each other.}
\\\hangindent1em
Proof:
$\ip{n_1'n_2'\cd|a_p[n_1n_2\cd]}$
vanishes unless $n_p'=0$, $n_p=1$, and $n_q'=n_q$ for $q\neq p$, in which case it equals $(-)^{n_1+\cd+n_{p-1}}$.
The same is true for 
$\ip{c_p[n_1'n_2'\cd]|n_1n_2\cd}$.
Therefore, $\ip{\Y|a_p\Y'}=\ip{c_p\Y|\Y'}$ for arbitrary linear combinations $\Y$ and $\Y'$ of the basis states, which implies $c_p=a_p\dg$ by the definition of adjoint.
\end{pro}

\begin{pro}
\label{particle-hole-commutation-rules}
\bmit{$[q,q']_+=\d_{q'q\dg}$.}
\textit{Particle-hole operators $q$ and $q'$ anticommute unless $q'=q\dg$, for which $[q,q\dg]_+=1$.}
\\\hangindent1em
Proof:
Let $q$ and $q'$ be arbitrary particle-hole operators acting on states $\y_p$ and $\y_{p'}$.
First, suppose $p\neq p'$.
Then
\begin{align*}
&&
&
  qq'\kt{\cd n_p\cd n_{p'}\cd}
=
  (-)^{n_p+\sum_{r=p+1}^{p'}n_r}\kt{\cd\ol{n_p}\cd\ol{n_{p'}}\cd}
\\
&&
 \text{and}\ \ 
&
  q'q\kt{\cd n_p\cd n_{p'}\cd}
=
  (-)^{\ol{n_p}+\sum_{r=p+1}^{p'}n_r}\kt{\cd\ol{n_p}\cd\ol{n_{p'}}\cd}
\end{align*}
where $\ol{n_p}$ and $\ol{n_{p'}}$ are the occupations after operating $q$ and $q'$ on the state.
Since $n_p$ and $\ol{n_p}$ differ by one, $qq'=-q'q$.
Now, consider the case $p=p'$ which implies $q'\in\{q,q\dg\}$.
If $q'=q$ then $qq'=-q'q=0$.
In the final case, $q'=q\dg$, we have either $a_p\dg a_p\kt{\cd n_p\cd}=\kt{\cd n_p\cd}$ and $a_pa_p\dg\kt{\cd n_p\cd}=0$ if $n_p=1$ or vice versa if $n_p=0$, so that $(qq\dg+q\dg q)=1$.
\end{pro}

\begin{rmk}
\label{fock-space-slater-determinants}
\bmit{Representing Slater determinants in terms of particle-hole operators.}
The Slater determinant $\F_{(p_1\cd p_n)}$ can be represented in $F(\mc{H})$ as
\begin{align}
&&
\label{fock-space-determinant}
  \kt{\F_{(p_1\cd p_n)}}
=
  a_{p_1}\dg\cd a_{p_n}\dg\kt{\vac}
\end{align}
which is equivalent to $\kt{(p_1\cd p_n)}$ for $p_1<\cd<p_n$, and is otherwise equivalent to $\e_\pi\kt{(p_{\pi(1)}\cd p_{\pi(n)})}$ for $\pi\in\mr{S}_n$ such that $p_{\pi(1)}<\cd<p_{\pi(n)}$.
The actions of $a_p$ and $a_p\dg$ on $\F_{(p_1\cd p_n)}$ are given by
\begin{align}
\label{determinant-annihilation}
&
  a_p\F_{(p_1\cd p_n)}
=
  (-)^{k-1}
  \F_{(p_1\cd \cancel{p_k}\cd p_n)}
\ \ \text{if $p=p_k\in(p_1\cd p_n)$}
&&
  a_p\F_{(p_1\cd p_n)}
=
  0
\ \ \text{if $p\notin(p_1\cd p_n)$}
\\
\label{determinant-creation}
&
  a_p\dg\F_{(p_1\cd p_n)}
=
  (-)^{k-1}
  \F_{(p_1\cd p_{k-1}pp_k\cd p_n)}
\ \ \text{if $p\notin(p_1\cd p_n)$}
&&
  a_p\dg\F_{(p_1\cd p_n)}
=
  0
\ \ \text{if $p\in(p_1\cd p_n)$}
\end{align}
which follows from equation \ref{fock-space-determinant} and \Cref{particle-hole-commutation-rules}.
\end{rmk}

\begin{dfn}
\label{normal-order}
\bmit{$\vac$-normal order.}
A string $q_1\etc q_n$ of particle-hole operators $q_i\in \{a_p\}\cup\{a_p\dg\}$ is in \textit{vacuum normal order} (or \textit{$\vac$-normal order}) when all of its creation operators lie to the left of all of its annihilation operators.
That is, a string of particle-hole operators is in $\vac$-normal order if it has the form $a_{p_1}\dg\cd a_{p_m}\dg a_{r_1}\cd a_{r_{m'}}$.
This guarantees that $\ip{\vac|q_1\etc q_n|\vac}=0$, i.e. its vacuum expectation value vanishes.
\end{dfn}

\begin{dfn}
\label{particle-number-conserving-operators}
\bmit{Particle-number conserving strings.}
A string $q_1\etc q_n$ is \textit{particle-number conserving} if it contains the same number of creation and annihilation operators, and so maps each $m$-particle subspace $\mb{A}(\mc{H}^{\otimes m})$ into itself.
\end{dfn}

\begin{dfn}
\bmit{Excitation operators.}
Excitation operators are particle-number conserving strings in normal order, i.e. strings which have have the form $a_{p_1}\dg\cd a_{p_m}\dg a_{q_m}\cd a_{q_1}$ ($=a_{p_1}\dg a_{q_1}\cd a_{p_m}\dg a_{q_m}$ if $\{p_i\}\cap\{q_i\}=\O$).
For a given reference determinant $\F$, excited determinants are given by $\F_{i_1\cd i_m}^{a_1\cd a_m}=a_{a_1}\dg\cd a_{a_m}\dg a_{i_m}\cd a_{i_1}\F$ where $i_1\cd i_m$ are occupied and $a_1\cd a_m$ are virtual indices with respect to $\F$.
\end{dfn}

\begin{rmk}
\label{fock-space-operators}
\bmit{The Fock space Hamiltonian.}
The electronic Hamiltonian can be expressed in $F(\mc{H})$ as
\begin{align*}
&&
  H_e
=
  \sum_{pq}
  h_{pq} a_p\dg a_q
+
  \fr{1}{2}
  \sum_{pqrs}
  \ip{pq|rs} a_p\dg a_q\dg a_sa_r
\end{align*}
where
$
  h_{pq}
=
  \ip{\y_p(1)|\op{h}(1)|\y_q(1)}
$
and
$
  \ip{pq|rs}
=
  \ip{\y_p(1)\y_q(2)|\op{g}(1,2)|\y_r(1)\y_s(2)}
$
are the one- and two-electron integrals.
\end{rmk}


\begin{dfn}
\bmit{One-particle and one-hole density matrices.}
The \textit{one-particle} and \textit{one-hole density matrices} of a state $\Y\in F(\mc{H})$ are given by $\g_{pq}\equiv\ip{\Y|a_q\dg a_p|\Y}$ and $\h_{pq}\equiv\ip{\Y|a_pa_q\dg|\Y}$, respectively.
\end{dfn}



\section{$\vac$-normal ordering}

\begin{dfn}
\bmit{$\vac$-normal ordering.}
The \textit{$\vac$-normal ordering} of a string $q_1\cd q_n$ of particle-hole operators is the mapping $q_1\cd q_n\mapsto\NO{q_1\etc q_n}\equiv\e_\pi q_{\pi(1)}\etc q_{\pi(n)}$ where $\pi\in\mr{S}_n$ is a permutation that places the string in normal order.
\end{dfn}

\begin{dfn}
\label{contraction}
\bmit{$\vac$-normal contraction.}
A pairwise \textit{$\vac$-normal contraction}, $\ctr{}{q}{_1}{q}{_2}q_1q_2$, of two particle-hole operators $q_1$ and $q_2$ is defined as $\ctr{}{q}{_1}{q}{_2}q_1q_2\equiv q_1q_2-\NO{q_1q_2}$.  This associates a scalar value with every pair in $\{a_p\}\cup\{a_p\dg\}$, of which the only non-trivial case is 
$\ctr{}{a}{_p}{a}{_q\dg}a_p a_q\dg=a_pa_q\dg-\NO{a_pa_q\dg}=a_pa_q\dg+a_q\dg a_p=\d_{pq}$.
All other \vac-normal contractions, including $\ctr{}{a}{_p\dg}{a}{_q}a_p\dg a_q$, are zero.
\end{dfn}

\begin{rmk}
\bmit{Non-vanishing \vac-normal contractions are elements of the \vac\ one-hole density matrix.}
Note that \vac-normal contractions can in general be identified with $\ctr{}{q}{_1}{q}{_2}q_1q_2=\ip{\vac|q_1q_2-\NO{q_1q_2}|\vac}=\ip{\vac|q_1q_2|\vac}$, since $\ip{\vac|\NO{q_1q_2}|\vac}=0$ and $\ctr{}{q}{_1}{q}{_2}q_1q_2$ is a scalar.
The only non-trivial contractions are those with an annihilation operator on the left and a creation operator on the right, which are elements of the one-hole density matrix $\ctr{}{a}{_p}{a}{_q\dg}a_pa_q\dg=\ip{\vac|a_pa_q\dg|\vac}=\h_{pq}=\d_{pq}$ of the vacuum.
\end{rmk}

\begin{dfn}
\bmit{$\vac$-normal ordering with contractions.}
If $q_1\cd q_n$ is a string of particle-hole operators, its \textit{$\vac$-normal ordering with contraction $\ctr{}{q}{_i}{q}{_j}q_iq_j$} ($i<j$) is  defined as $\NO{\ctr{q_1\etc}{q}{_i\etc}{q}{_j\etc q_n}q_1\etc q_i\etc q_j\etc q_n}\equiv (-)^{j-i-1}\ctr{}{q}{_i}{q}{_j}q_iq_j\NO{q_1\cd\cancel{q_i}\cd\cancel{q_j}\cd q_n}$ where the phase factor corresponds to the signature of the permutation required to bring $q_i$ and $q_j$ together.
$\vac$-normal-ordered products with multiple contractions are defined analogously.
\end{dfn}

\begin{ntt}
\label{contraction-notation}
Given a string $Q=q_1\cd q_n$ of particle-hole operators, let $\NO{Q(\ctr{}{q}{_i}{q}{_j}q_iq_j)}$ denote $\NO{\ctr{q_1\etc}{q}{_i\etc}{q}{_j\etc q_n}q_1\etc q_i\etc q_j\etc q_n}$.
This is unambiguous for $i<j$.
Let $\NO{\ol{Q}}$ denote the sum of all unique single, double, triple, etc. contractions, i.e.
\begin{align*}
&&
  \NO{\ol{Q}}
\equiv
  \sum_{k=1}^{\floor{\sfrac{n}{2}}}
  \sum^{\mr{Ctr}_k(Q)}_{\miniar{i_1\cd i_k\\j_1\cd j_k}}
  \NO{Q(
    \ctr{}{q}{_{i_1}}{q}{_{j_1}}q_{i_1}q_{j_1}
    \cd
    \ctr{}{q}{_{i_k}}{q}{_{j_k}}q_{i_k}q_{j_k}
  )}
\end{align*}
where $\mr{Ctr}_k(Q)$ runs over the unique sets of $k$ pairs of operator indices in $Q$ and $\floor{}$ is the floor function.
Let $\NO{\ol{\ol{Q}}}$ denote the sum of all \textit{complete contractions} of $Q$, i.e. the terms in $\NO{\ol{Q}}$ in which all operators in the product are involved in a contraction.
For a pair of particle-hole operator strings $Q$ and $Q'$, let $\NO{\ctr{}{Q}{}{Q}{'}QQ'}$ denote the sum of all \textit{cross contractions} between $Q$ and $Q'$ in $\NO{\ol{QQ'}}$, i.e. omitting any \textit{internal contractions} in $Q$ or $Q'$.
Finally, let $\NO{\ccr{}{Q}{}{Q}{'}QQ'}$ denote the sum of \textit{complete cross contractions} between $Q$ and $Q'$ in $\NO{\ctr{}{Q}{}{Q}{'}QQ'}$, i.e. those in which every operator in $Q$ is contracted with an operator in $Q'$ and no uncontracted operators are left over.
\end{ntt}


\section{Wick's theorem}

\begin{lem}
\label{wick-lem}
\bmit{$\NO{Q}q=\NO{Qq}+\sum_k\NO{\ctr{Q(}{q}{_k)}{q}{}Q(q_k)q}$}
\\\hangindent1em
Proof:
We can assume without loss of generality that $Q$ is already in normal order so that $\NO{Q}=Q$.
Let $n$ be the number of operators in $Q$.
If $q$ is an annihilation operator then $\NO{Qq}=Qq$ and all of the contractions between $Q$ an $q$ vanish (see \Cref{contraction}), so the statement is trivially true.
If $q$ is a creation operator then $\NO{Qq}=(-)^nqQ$, and
\begin{align*}
  Qq
=
  (-)^nqQ
+
  \sum_{k=1}^n
  (-)^{n-k}
  q_1\etc [q_k,q]_+\etc q_n
=
  \NO{Qq}
+
  \sum_{k=1}^n
  \NO{\ctr{Q(}{q}{_k)}{q}{}Q(q_k)q}
\end{align*}
using the pull-through relation (see \Cref{pull-through}), the fact that $\NO{\ctr{Q(}{q}{_k)}{q}{}Q(q_k)q}=(-)^{n-k}q_1\etc \ctr{}{q}{_k}{}{q}q_kq\etc q_n$, and the fact that $\ctr{}{q}{_k}{q}{}q_kq = [q_k,q]_+$ when $q$ is a creation operator (see again \Cref{contraction}).
\end{lem}

\begin{thm}
\label{wick-thm}
\bmit{Wick's Theorem, $Q=\NO{Q}+\NO{\ol{Q}}$ (time-independent).}
\textit{Any string $Q$ of particle-hole operators is equal to $\NO{Q}+\NO{\ol{Q}}$, its normal-ordered form plus the sum of all possible contractions.}
\\\hangindent1em\hangafter2
Proof:
Let $Q=q_1\etc q_n$.
The result holds for $n=2$ because $q_1q_2=\NO{q_1q_2}+\ctr{}{q}{_1}{q}{_2}q_1q_2$ follows immediately from \Cref{contraction}.
Now, assume it holds for $Q$ with $n$ operators and consider $Qq$.
By our inductive assumption $Qq=\NO{Q}q+\NO{\ol{Q}}q$.
Applying \Cref{wick-lem} to $\NO{Q}q$, we find
$
  \NO{Q}q
=
  \NO{Qq}
+
  \sum_i
  \NO{\ctr{Q(}{q}{_i)}{q}{}Q(q_i)q}
$.
Expanding $\NO{\ol{Q}}q$ and applying \Cref{wick-lem} to each term gives
\begin{align*}
  \sum_{k=1}^{\floor{\sfrac{n}{2}}}
  \sum^{\mr{Ctr}_k(Q)}_{\miniar{i_1\cd i_k\\j_1\cd j_k}}
  \NO{Q(
    \ctr{}{q}{_{i_1}}{q}{_{j_1}}q_{i_1}q_{j_1}
    \cd
    \ctr{}{q}{_{i_k}}{q}{_{j_k}}q_{i_k}q_{j_k}
  )}
  q
&=
  \sum_{k=1}^{\floor{\sfrac{n}{2}}}
  \sum^{\mr{Ctr}_k(Q)}_{\miniar{i_1\cd i_k\\j_1\cd j_k}}
    \NO{Q(
      \ctr{}{q}{_{i_1}}{q}{_{j_1}}q_{i_1}q_{j_1}
      \cd
      \ctr{}{q}{_{i_k}}{q}{_{j_k}}q_{i_k}q_{j_k}
    )
    q}
  +
  \sum_{k=1}^{\floor{\sfrac{n}{2}}}
  \sum^{\mr{Ctr}_k(Q)}_{\miniar{i_1\cd i_k\\j_1\cd j_k}}
    \sum_{i\notin\{{\miniar{i_1\cd i_k\\j_1\cd j_k}}\}}
    \NO{Q(
      \ctr{}{q}{_{i_1}}{q}{_{j_1}}q_{i_1}q_{j_1}
      \cd
      \ctr{}{q}{_{i_k}}{q}{_{j_k}}q_{i_k}q_{j_k}
      \ctr{}{q}{_i)}{q}{}
      q_i)q
    }
\\&=
  \sum^{\mr{Ctr}_1(Q)}_{\miniar{i_1\\j_1}}
    \NO{Q(
      \ctr{}{q}{_{i_1}}{q}{_{j_1}}q_{i_1}q_{j_1}
    )
    q}
  +
  \sum_{k=2}^{\floor{\sfrac{(n+1)}{2}}}
  \sum^{\mr{Ctr}_k(Qq)}_{\miniar{i_1\cd i_k\\j_1\cd j_k}}
    \NO{Qq(
      \ctr{}{q}{_{i_1}}{q}{_{j_1}}q_{i_1}q_{j_1}
      \cd
      \ctr{}{q}{_{i_k}}{q}{_{j_k}}q_{i_k}q_{j_k}
      )
    }
\end{align*}
and, combining these results, we find
\begin{align*}
  Qq
=
  \NO{Q}q
+
  \NO{\ol{Q}}q
=&\
  \NO{Qq}
+
  \sum_i
  \NO{\ctr{Q(}{q}{_i)}{q}{}Q(q_i)q}
+
  \sum^{\mr{Ctr}_1(Q)}_{\miniar{i_1\\j_1}}
    \NO{Q(
      \ctr{}{q}{_{i_1}}{q}{_{j_1}}q_{i_1}q_{j_1}
    )
    q}
  +
  \sum_{k=2}^{\floor{\sfrac{(n+1)}{2}}}
  \sum^{\mr{Ctr}_k(Qq)}_{\miniar{i_1\cd i_k\\j_1\cd j_k}}
    \NO{Qq(
      \ctr{}{q}{_{i_1}}{q}{_{j_1}}q_{i_1}q_{j_1}
      \cd
      \ctr{}{q}{_{i_k}}{q}{_{j_k}}q_{i_k}q_{j_k}
      )
    }
\\=&\
  \NO{Qq}
+
  \sum_{k=1}^{\floor{\sfrac{(n+1)}{2}}}
  \sum^{\mr{Ctr}_k(Qq)}_{\miniar{i_1\cd i_k\\j_1\cd j_k}}
    \NO{Qq(
      \ctr{}{q}{_{i_1}}{q}{_{j_1}}q_{i_1}q_{j_1}
      \cd
      \ctr{}{q}{_{i_k}}{q}{_{j_k}}q_{i_k}q_{j_k}
      )
    }
\end{align*}
which is $\NO{Qq}+\NO{\ol{Qq}}$.
So if the statement holds for strings of length $n$ it must also hold for strings of length $n+1$ and, by induction, the theorem holds for $Q$ of arbitary length.
\end{thm}

\begin{cor}
\label{wick-product}
\bmit{Wick's Theorem for operator products.}
\textit{
Given a pair $Q, Q'$ of particle-hole operator strings already in normal order, the product of their normal orderings is given by $\NO{Q}\NO{Q'}=\NO{QQ'}+\NO{\ctr{}{Q}{}{Q}{'}QQ'}$.}
\\\hangindent1em\hangafter2
Proof:
By Wick's theorem $\NO{Q}\NO{Q'}=\NO{QQ'}+\NO{\ol{\pr{\NO{Q}\NO{Q'}}}}$.
Since $\NO{Q}$ and $\NO{Q'}$ are in normal order their internal contractions in \NO{\ol{\pr{\NO{Q}\NO{Q'}}}} vanish, which implies $\NO{\ol{\pr{\NO{Q}\NO{Q'}}}}=\NO{\ctr{}{Q}{}{Q}{'}QQ'}$.
\end{cor}

\begin{cor}
\label{wick-vacuum-expectation}
\bmit{$\ip{\vac|Q|\vac}=\NO{\ol{\ol{Q}}}$.}
\textit{The vacuum expectation value of a string of particle-hole operators equals the sum of its complete contractions.}
\\\hangindent1em\hangafter2
Proof:
From Wick's theorem (\Cref{wick-thm}) we have that $Q=\NO{Q}+\NO{\ol{Q}}$.  Taking the vacuum expectation value of both sides gives $\ip{\vac|Q|\vac}=\NO{\ol{\ol{Q}}}$ since all terms with incompletely contracted operators have vanishing \vac\ expectation values.
\end{cor}

\begin{pro}
\bmit{Sign rule for completely contracted products.}
\textit{The phase factor of a completely contracted product is given by $(-)^{c}$ where $c$ is the number of crossings between contraction lines.}
\\\hangindent1em\hangafter2
Proof: Let $\e_\pi\NO{\ctr{}{q}{_{\pi(1)}}{q}{_{\pi(2)}}
                     q_{\pi(1)}q_{\pi(2)}\cd
                     \ctr{}{q}{_{\pi(2n-1)}}{q}{_{\pi(2n)}}
                     q_{\pi(2n-1)}q_{\pi(2n)}}$
be the disentangled form of a complete contraction of $q_1\cd q_{2n}$.
The phase factor for the contraction, $\e_\pi$, is equal to the signature of the disentangling permutation, which is equal to the signature of its inverse (``re-entangling'') permutation.
Consider re-entangling the product by transpositions of adjacent operators, noting that we only need transpositions involving operator pairs which are not contracted with each other.
Since every operator has a contraction line overhead, each transposition changes the number of crossings by exactly $\pm1$.
Therefore, $\e_\pi=(-)^c$ where $c$ is the number of crossings.
\end{pro}


\section{$\F$-normal ordering}

\begin{rmk}
\label{hp-isomorphism}
\bmit{Hole-particle isomorphism.}
Let $\F$, corresponding to the occupation vector $\kt{\underset{\text{$n$ times}}{1\cd 1}000\cd}$, be a reference determinant and consider the mapping $F(\mc{H})\rightarrow F(\mc{H})$ given by inverting the bits occupied in $\F$:
\begin{align*}
&&
  \kt{n_1\cd n_n n_{n+1}n_{n+2}\cd}
\mapsto
  \kt{\ol{n}_1\cd \ol{n}_n n_{n+1}n_{n+2}\cd}
\ \ \text{where} \ \ 
  \ol{n}_i
=
  1
-
  n_i
  \ .
\end{align*}
This mapping is invertible, so it defines an isomorphism.
Physically, this corresponds to viewing the first $n$ states as \textit{hole states} rather than as \textit{particle states}.
Collectively, these are referred to as \textit{quasiparticle states}.
Under this isomorphism, $\kt{\vac}\mapsto\kt{\underset{\text{$n$ times}}{\ol{1}\cd \ol{1}}000\cd}$ becomes a state of $n$ holes and $\kt{\F}\mapsto\kt{\underset{\text{$n$ times}}{\ol{0}\cd \ol{0}}000\cd}$ becomes the \textit{quasiparticle vacuum}, in which all hole and particle states are unoccupied.
\end{rmk}

\begin{dfn}
\setlength{\belowdisplayskip}{0pt}
\setlength{\belowdisplayshortskip}{0pt}
\label{qparticle-creation-annihilation-ops}
\bmit{Quasiparticle creation and annihilation operators.}
Let $\{b_p\}\cup\{b_p\dg\}$ be the \textit{quasiparticle creation and annihilation operators} that result from applying the isomorphism discussed in \Cref{hp-isomorphism}.
These are related to the original set $\{a_p\}\cup\{a_p\dg\}$ of \textit{particle creation and annihilation operators} via
\begin{align*}
&&
  a_i
\mapsto
  b_i\dg
&&
  a_i\dg
\mapsto
  b_i
&&
  a_a
\mapsto
  b_a
&&
  a_a\dg
\mapsto
  b_a\dg
\end{align*}
where $i$ and $a$ are occupied and virtual indices with respect to $\F$.
$\{a_i\dg\}\cup\{a_a\}\mapsto\{b_p\}$ are therefore \textit{quasiparticle annihilation operators} and $\{a_i\}\cup\{a_a\dg\}\mapsto\{b_p\dg\}$ are \textit{quasiparticle creation operators}.
\end{dfn}

\begin{dfn}
\bmit{$\F$-normal order.}
A string $q_1\cd q_n$ of particle-hole operators $q_i\in\{a_p\}\cup\{a_p\dg\}$ is in \textit{$\F$-normal order} when all of its quasiparticle creation operators lie to the left of all of its quasiparticle annihilation operators.
That is, it must map to a string of the form $b_{p_1}\dg\cd b_{p_m}\dg b_{r_1}\cd b_{r_{m'}}$ under the hole-particle isomorphism (\Cref{hp-isomorphism}) that makes $\F$ the quasiparticle vacuum.
This quarantees that $\ip{\F|q_1\cd q_n|\F}=0$, i.e. its $\F$ expectation value vanishes.
\end{dfn}

\begin{dfn}
\bmit{$\F$-normal ordering.}
The \textit{$\F$-normal ordering} of a string $q_1\cd q_n$ of particle-hole operators is the mapping $q_1\cd q_n\mapsto\GNO{q_1\cd q_n}=\e_\pi q_{\pi(1)}\cd q_{\pi(n)}$ where $\pi\in\mr{S}_n$ is a permutation that places the string in $\F$-normal order.
\end{dfn}


\begin{dfn}
\bmit{$\F$-normal contraction.}
A pairwise \textit{$\F$-normal contraction} $\ctr{}{q}{_1}{q}{_2}q_1q_2$ is defined as $\ctr{}{q}{_1}{q}{_2}q_1q_2\equiv q_1q_2-\GNO{q_1q_2}$.
For the quasiparticle operators of \Cref{qparticle-creation-annihilation-ops}, the only non-vanishing contraction is $\ctr{}{b}{_p}{b}{_q\dg}b_pb_q\dg=\d_{pq}$, which shows that $\ctr[0.5]{}{a}{_i\dg}{a}{_j}a_i\dg a_j=\d_{ij}$ and $\ctr{}{a}{_a}{a}{_b\dg}a_aa_b\dg=\d_{ab}$ are the non-vanishing $\F$-normal contractions of the orignal particle-hole operators.
\end{dfn}

\begin{rmk}
\label{phi-opdm-ohdm}
\bmit{Non-vanishing contractions are elements of the $\F$ one-particle and one-hole density matrices.}
$\F$-normal contractions can in general be identified with $\ctr{}{q}{_1}{q}{_2}q_1q_2=\ip{\F|q_1q_2-\GNO{q_1q_2}|\F}=\ip{\F|q_1q_2|\F}$.
Therefore, the non-vanishing contractions are elements of the $\F$ one-particle and one-hole density matrices
\begin{align*}
&&
  \ctr[0.5]{}{a}{_p\dg}{a}{_q}a_p\dg a_q
=
  \ip{\F|a_p\dg a_q|\F}
=
  \g_{qp}
&&
  \ctr{}{a}{_p}{a}{_q\dg}a_p a_q\dg
=
  \ip{\F|a_p a_q\dg|\F}
=
  \h_{pq}
\end{align*}
depending whether the creation operator is on the right or left in the contraction.
The one-particle density matrix of $\F$ is given by $\g_{pq}=0$ unless $p=q=i$ where $i$ is an occupied index, in which case $\g_{ii}=1$.
The one-hole density matrix of $\F$ is given by $\h_{pq}=0$ unless $p=q=a$ where $a$ is a virtual index, in which case $\h_{aa}=1$.
\end{rmk}

\begin{dfn}
\bmit{$\F$-normal ordering with contractions.}
If $q_1\cd q_n$ is a string of particle-hole operators, its \textit{$\F$-normal ordering with contraction $\ctr{}{q}{_i}{q}{_j}q_iq_j$} ($i<j$) is  defined as $\GNO{\ctr{q_1\etc}{q}{_i\etc}{q}{_j\etc q_n}q_1\etc q_i\etc q_j\etc q_n}\equiv (-)^{j-i-1}\ctr{}{q}{_i}{q}{_j}q_iq_j\GNO{q_1\cd\cancel{q_i}\cd\cancel{q_j}\cd q_n}$.
\end{dfn}

\begin{rmk}
\bmit{Wick's theorem for $\F$-normal ordering.}
The isomorphism described in \Cref{hp-isomorphism} maps $\F$-normal ordered products with $\F$-normal contractions into $\vac$-normal ordered products with $\vac$-normal contractions.
Therefore, Wick's theorem still holds as
$
  Q
=
  \GNO{Q}
+
  \GNO{\ol{Q}}
$
with non-vanishing contractions
$
  \ctr[0.5]{}{a}{_p\dg}{a}{_q}a_p\dg a_q
=
  \g_{qp}
$
and
$
  \ctr{}{a}{_p}{a}{_q\dg}a_p a_q\dg
=
  \h_{pq}
$.
The corollaries $\GNO{Q}\GNO{Q'}=\GNO{QQ'}+\GNO{\ctr{}{Q}{}{Q}{'}QQ'}$ and $\ip{\F|Q|\F}=\GNO{\ol{\ol{Q}}}$ also still hold, since they follow directly from Wick's theorem.
\end{rmk}

\begin{rmk}
\label{excitation-operator-wick-expansion}
As an example, consider the $\Phi$-normal Wick expansion of $a_p\dg a_q$ and $a_p\dg a_q\dg a_sa_r$:
\begin{align*}
  a_p\dg a_q
=&\
  \GNO{a_p\dg a_q}
+
  \GNO{\ctr{}{a}{_p\dg}{a}{_q}a_p\dg a_q}
=
  \GNO{a_p\dg a_q}
+
  \g_{qp}
\\
  a_p\dg a_q\dg a_sa_r
=&\
  \GNO{a_p\dg a_q\dg a_sa_r}
+
  \GNO{
  \ctr[0.5]{}{a}{_p\dg a_q\dg}{a}{_sa_r}
  a_p\dg a_q\dg a_sa_r
  }
+
  \GNO{
  \ctr[0.5]{}{a}{_p\dg a_q\dg a_s}{a}{_r}
  a_p\dg a_q\dg a_sa_r
  }
+
  \GNO{
  \ctr[0.5]{a_p\dg}{a}{_q\dg}{a}{_sa_r}
  a_p\dg a_q\dg a_sa_r
  }
+
  \GNO{
  \ctr[0.5]{a_p\dg}{a}{_q\dg a_s}{a}{_r}
  a_p\dg a_q\dg a_sa_r
  }
+
  \GNO{
  \ctr[2]{}{a}{_p\dg a_q\dg a_s}{a}{_r}
  \ctr[0.5]{a_p\dg}{a}{_q\dg}{a}{_sa_r}
  a_p\dg a_q\dg a_sa_r
  }
+
  \GNO{
  \ctr[2]{a_p\dg}{a}{_q\dg a_s}{a}{_r}
  \ctr[0.5]{}{a}{_p\dg a_q\dg}{a}{_sa_r}
  a_p\dg a_q\dg a_sa_r
  }
\\=&\
  \GNO{a_p\dg a_q\dg a_sa_r}
-
  \g_{sp}
  \GNO{a_q\dg a_r}
+
  \g_{rp}
  \GNO{a_q\dg a_s}
+
  \g_{sq}
  \GNO{a_p\dg a_r}
-
  \g_{rq}
  \GNO{a_p\dg a_s}
+
  \g_{rp}\g_{sq}
-
  \g_{sp}\g_{rq}
\ \ .
\end{align*}
In general, an $m$-tuple excitation operator $a_{p_1}\dg\cd a_{p_m}\dg a_{q_m}\cd a_{q_1}$ can be expressed as a linear combination of $\F$-normal ordered $n$-tuple excitation operators $\GNO{a_{r_1}\dg\cd a_{r_n}\dg a_{s_n}\cd a_{s_1}}$ from $n=0$ to $n=m$.
\end{rmk}



\section{Kutzelnigg-Mukherjee tensor notation}

\begin{ntt}
\label{kutzelnigg-mukherjee-vacuum}
\bmit{Kutzelnigg-Mukherjee tensor notation.}
In \textit{Kutzelnigg-Mukherjee tensor notation} (\textit{KM notation}), indices which are covariant with respect to the spin-orbital basis $\{\y_p\}$ are written as superscripts, whereas contravariant indices are written as subscripts.
Note that this reverses the usual convention in multilinear algebra.
Furthermore, the Einstein summation convention is adopted, in which any pair of matching upper and lower indices in a product is implicitly summed over its index range.
Indices $i,j,k,l,\cd$ count over occupied orbitals $\{\y_i\in\F\}$, indices $a,b,c,d,\cd$ count over virtual orbitals $\{\y_a\notin\F\}$, and indices $p,q,r,s,\cd$ count over the full spin-orbital basis $\{\y_p\in\mc{H}\}$.
\end{ntt}

\begin{ntt}
\bmit{KM notation for particle-hole operators.}
Annihilation operators ($a_p$) are contravariant and creation operators ($a_p\dg$) operators are covariant to the spin-orbital basis, so these are written as $a_p$ and $a^p$, respectively.
\end{ntt}

\begin{ntt}
\bmit{KM notation for one-hole and one-particle density matrices.}
The one-particle and one-hole density matrices of a state $\Y$ are written as $\g_q^p=\ip{\Y|a^pa_q|\Y}$ and $\h_p^q=\ip{\Y|a_pa^q|\Y}$, respectively.
\end{ntt}

\begin{ntt}
\bmit{KM notation for excitation operators.}
Excitation operators are given the following concise notation.
\begin{align*}
&&
  a_{q_1\cd q_m}^{p_1\cd p_m}
\equiv
  a^{p_1}\cd a^{p_m}a_{q_m}\cd a_{q_1}
\end{align*}
$\Phi$-normal-ordered excitation operators are denoted with a tilde:
$
  \tl{a}_{q_1\cd q_m}^{p_1\cd p_m}
\equiv
  \GNO{a_{q_1\cd q_m}^{p_1\cd p_m}}
$.
\end{ntt}

\begin{ntt}
\bmit{KM notation for $H_e$.}
The electronic Hamiltonian is expressed in KM notation as
\begin{align*}
&&
  H_e
=
  h_p^q
  a_q^p
+
  \tfrac{1}{2}
  g_{pq}^{rs}
  a_{rs}^{pq}
\sp \text{or}\sp
  H_e
=
  h_p^q
  a_q^p
+
  \tfrac{1}{4}
  \ol{g}_{pq}^{rs}
  a_{rs}^{pq}
\end{align*}
where $h_p^q\equiv\ip{\y_p(1)|\op{h}(1)|\y_q(1)}$ are the one-electron integrals, 
$g_{pq}^{rs}\equiv\ip{\y_p(1)\y_q(2)|\op{g}(1,2)|\y_r(1)\y_s(2)}$
are the two-electron integrals, and $\ol{g}_{pq}^{rs}\equiv g_{pq}^{rs}-g_{pq}^{sr}$ are the antisymmetrized two-electron integrals.
\end{ntt}

\begin{rmk}
\label{excitation-operator-phase}
Note that for normal-ordered products of excitation operators, the following rearrangements are valid.
\begin{align*}
&&
  \NO{a_{q_1\cd q_m}^{p_1\cd p_m}
      a_{s_1\cd s_n}^{r_1\cd r_n}}
=
  \NO{a_{q_1}^{p_1}\cd a_{p_m}^{p_m}
      a_{s_1}^{r_1}\cd a_{s_n}^{r_n}}
=
  a_{q_1\cd q_ms_1\cd s_n}^{p_1\cd p_mr_1\cd r_n}
\\
&&
  \GNO{\tl{a}_{q_1\cd q_m}^{p_1\cd p_m}
       \tl{a}_{s_1\cd s_n}^{r_1\cd r_n}}
=
  \GNO{\tl{a}_{q_1}^{p_1}\cd \tl{a}_{p_m}^{p_m}
       \tl{a}_{s_1}^{r_1}\cd \tl{a}_{s_n}^{r_n}}
=
  \tl{a}_{q_1\cd q_ms_1\cd s_n}^{p_1\cd p_mr_1\cd r_n}
\end{align*}
Furthermore, every one of these expressions is antisymmetric in its upper and lower indices -- for example,
$
  \NO{a_{q}^{p}a_{tu}^{rs}}
=
-
  \NO{a_{q}^{r}a_{tu}^{ps}}
=
  \NO{a_{t}^{r}a_{qu}^{ps}}
$.
The phase factor for permutations of upper or lower indices is $(-)^t$ where $t$ is the number of transpositions required to restore the original pairing ($\miniar{p_i\\q_i}$ and $\miniar{r_i\\s_i}$) of upper and lower indices.
When the pairs are permuted together, no change in sign appears: $a_{q_1\cd q_m}^{p_1\cd p_m}=a_{q_{\pi(1)}\cd q_{\pi(m)}}^{p_{\pi(1)}\cd p_{\pi(m)}}$ for all $\pi\in\mr{S}_m$.
\end{rmk}

\begin{ntt}
\label{km-contraction-notation}
\bmit{Notation for contractions.}
It is convenient here to shift to the following notation for contractions
\begin{align*}
&&
  a_{p^\hole}a^{q^\hole}
\equiv
  \ctr{}{a}{_p}{a}{^q}a_pa^q
&&
  a^{q^\hole}a_{p^\hole}
\equiv
-
  \ctr{}{a}{_p}{a}{^q}a_pa^q
&&
  a^{p^\ptcl}a_{q^\ptcl}
\equiv
  \ctr{}{a}{^p}{a}{_q}a^pa_q
&&
  a_{q^\ptcl}a^{p^\ptcl}
\equiv
-
  \ctr{}{a}{^p}{a}{_q}a^pa_q
\end{align*}
where $\hole$ indicates an \textit{$\h$-contraction} and $\ptcl$ indicates a \textit{$\g$-contraction} between operators with the same number of these symbols next to their indices.
Note that $a_{p^\hole}^{q^\hole}=-\h_p^q$ ($=-\d_p^q$ for a $\vac$-normal contraction) and $a_{q^\ptcl}^{p^\ptcl}=\g_q^p$ ($=0$ for a $\vac$-normal contraction).
This notation allows for contracted operators to be permuted arbitrarily with the other operators in a normal-ordered string, in contrast to $\ctr{}{a}{_p}{a}{^q}a_pa^q$ and $\ctr{}{a}{^p}{a}{_q}a^pa_q$ which are neither antisymmetric nor symmetric to interchange of their arguments.
\end{ntt}

\begin{ntt}
\label{index-permutation-operator}
\bmit{Index permutation operators.}
Let $\op{P}^{(P_1/\cd/P_M)}_{(Q_1/\cd/Q_N)}$, where $P_i$ and $Q_i$ are sets of indices, denote an \textit{index permutation operator} which yields a sum over all permutations of the indices in $P_1\cd P_M$ and $Q_1\cd Q_M$ weighted by the permutation phases, omitting any permutations that involve transpositions within one of the blocks, $P_i$ or $Q_i$.
\end{ntt}


\begin{rmk}
\label{km-notation-example}
Repeating the examples from \Cref{excitation-operator-wick-expansion} using KM notation, we write
\begin{align*}
&&
  a_q^p
=&\
  \GNO{a_q^p}
+
  \GNO{a_{q^\ptcl}^{p^\ptcl}}
&&=
  \tl{a}_q^p
+
  \g_q^p
\\
&&
  a_{rs}^{pq}
=&\
  \GNO{a_{rs}^{pq}}
+
  \GNO{a_{r^\ptcl s}^{p^\ptcl q}}
+
  \GNO{a_{rs^\ptcl}^{p^\ptcl q}}
+
  \GNO{a_{r^\ptcl s}^{pq^\ptcl}}
+
  \GNO{a_{rs^\ptcl}^{pq^\ptcl}}
+
  \GNO{a_{r^\ptcl s^{\ptcl\ptcl}}^{p^\ptcl q^{\ptcl\ptcl}}}
+
  \GNO{a_{r^{\ptcl\ptcl} s^\ptcl}^{p^\ptcl q^{\ptcl\ptcl}}}
&&=
  \tl{a}_{rs}^{pq}
+
  \op{P}_{(r/s)}^{(p/q)}
  \tl{a}_s^q\g_r^p
+
  \op{P}_{(r/s)}
  \g_r^p\g_s^q
\end{align*}
where $\op{P}_{(r/s)}$ acts as $\op{P}_{(r/s)}f_{rs} = f_{rs} - f_{sr}$ and $\op{P}_{(r/s)}^{(p/q)}$ acts as $\op{P}_{(r/s)}^{(p/q)}f_{rs}^{pq}=\op{P}^{(p/q)}\op{P}_{(r/s)}f_{rs}^{pq}=f_{rs}^{pq}-f_{sr}^{pq}-f_{rs}^{qp}+f_{sr}^{qp}$.
More complicated permutation operators appear in the $\F$-normal Wick expansions of higher excitation operators, such as
\begin{align*}
&&
  a_{stu}^{pqr}
=&\
  \tl{a}_{stu}^{pqr}
+
  \op{P}_{(s/tu)}^{(p/qr)}
  \g_s^p\tl{a}_{tu}^{qr}
+
  \op{P}_{(s/t/u)}^{(pq/r)}
  \g_s^p
  \g_r^q
  \tl{a}_u^r
+
  \op{P}_{(s/t/u)}
  \g_s^p
  \g_t^q
  \g_u^r
\end{align*}
Where $\op{P}_{(p/q/r)}$ sums over all permutations of $pqr$: $\op{P}_{(p/q/r)}f_{pqr}=f_{pqr}-f_{prq}-f_{qpr}+f_{qrp}+f_{rpq}-f_{rqp}$, whereas $\op{P}_{(p/qr)}$ omits any permutations that involve transpose $q$ and $r$: $\op{P}_{(p/qr)}f_{pqr}=f_{pqr}-f_{qpr}-f_{rqp}$.
\end{rmk}

\begin{rmk}
\label{phi-normal-expansion-of-hamiltonian}
The Wick expansions for $a_q^p$ and $a_{rs}^{pq}$ can be used to write $H_e=h_p^qa_q^p+g_{pq}^{rs}a_{rs}^{pq}$ as
\begin{align*}
&&
  H_e
=
  \ip{\F|H_e|\F}
+
  f_p^q\tl{a}_q^p
+
  \tfrac{1}{4}
  \ol{g}_{pq}^{rs}
  \tl{a}_{rs}^{pq}
\end{align*}
where $\ip{\F|H_e|\F}=h_p^q\g_q^p+\frac{1}{2}\ol{g}_{pq}^{rs}\g_r^p\g_s^q=h_i^i+\frac{1}{2}\ol{g}_{ij}^{ij}$ is the Hartree-Fock energy expression and $f_p^q=h_p^q+\ol{g}_{pr}^{qs}\g_s^r=h_p^q+\ol{g}_{pi}^{qi}$ is the spin-orbital Fock matrix.
\end{rmk}

\begin{rmk}
\bmit{Example: Derivation of Slater-Condon rules in KM notation.}
Noting $a_{ijk\cd}^{abc\cd}=\tl{a}_{ijk\cd}^{abc\cd}$ since these excitation operators consist of only quasiparticle creation operators, Hamiltonian matrix elements $\ip{\F|H_e|\F_{ijk\cd}^{abc\cd}}$ can be evaluated from $\ip{\F|\tl{a}_q^p\tl{a}_{ijk\cd}^{abc\cd}|\F}$ and $\ip{\F|\tl{a}_{rs}^{pq}\tl{a}_{ijk\cd}^{abc\cd}|\F}$.
According to Wick's theorem, $\ip{\F|\tl{a}_q^p\tl{a}_{ijk\cd}^{abc\cd}|\F}$ is equal to the complete cross-contractions between $\tl{a}_q^p$ and $\tl{a}_{ijk\cd}^{abc\cd}$, so that these matrix elements are as follows.
\begin{align*}
&
  \ip{\F|\tl{a}_q^p|\F}
=
  0
\\
&
  \ip{\F|\tl{a}_q^p\tl{a}_i^a|\F}
=
  \GNO{a_{q^\hole}^{p^\ptcl}a_{i^\ptcl}^{a^\hole}}
=
  \g_i^p\h_q^a
\\
&
  \ip{\F|\tl{a}_q^p\tl{a}_{ij}^{ab}|\F}
=
  0
\\
\intertext{Beyond single excitations, the matrix elements vanish because there are no possible complete cross-contractions.
Similarly, $\ip{\F|\tl{a}_{rs}^{pq}\tl{a}_{ijk\cd}^{abc\cd}|\F}$ is equal to the sum over complete cross-contractions between $\tl{a}_{rs}^{pq}$ and $\tl{a}_{ijk\cd}^{abc\cd}$.}
&
  \ip{\F|\tl{a}_{rs}^{pq}|\F}
=
  0
\\
&
  \ip{\F|\tl{a}_{rs}^{pq}\tl{a}_i^a|\F}
=
  0
\\
&
  \ip{\F|\tl{a}_{rs}^{pq}\tl{a}_{ij}^{ab}|\F}
=
  \GNO{a_{r^{\hole}s^{\hole\hole}}^{p^{\ptcl}q^{\ptcl\ptcl}}a_{i^{\ptcl}j^{\ptcl\ptcl}}^{a^{\hole}b^{\hole\hole}}}
+
  \GNO{a_{r^{\hole}s^{\hole\hole}}^{p^{\ptcl}q^{\ptcl\ptcl}}a_{i^{\ptcl}j^{\ptcl\ptcl}}^{a^{\hole\hole}b^{\hole}}}
+
  \GNO{a_{r^{\hole}s^{\hole\hole}}^{p^{\ptcl}q^{\ptcl\ptcl}}a_{i^{\ptcl\ptcl}j^{\ptcl}}^{a^{\hole}b^{\hole\hole}}}
+
  \GNO{a_{r^{\hole}s^{\hole\hole}}^{p^{\ptcl}q^{\ptcl\ptcl}}a_{i^{\ptcl\ptcl}j^{\ptcl}}^{a^{\hole\hole}b^{\hole}}}
=
  \op{P}^{(a/b)}_{(i/j)}
  \g^p_i\g^q_j\h_r^a\h_s^b
\\
&
  \ip{\F|\tl{a}_{rs}^{pq}\tl{a}_{ijk\cd}^{abc\cd}|\F}
=
  0
\end{align*}
Beyond double excitations, no complete cross-contractions are possible and the matrix elements vanish.
Using these results, the first Slater-Condon rule becomes tautological: $\ip{\F|H_e|\F}+f_p^q\ip{\F|\tl{a}_q^p|\F}+\tfrac{1}{4}\ol{g}_{pq}^{rs}\ip{\F|\tl{a}_{rs}^{pq}|\F}=\ip{\F|H_e|\F}$.
The remaining Slater-Condon rules take the following form.
\begin{align*}
&&
  \ip{\F|H_e|\F_i^a}
=
  f_p^q\g_i^p\h_q^a
=
  f_a^i
&&
  \ip{\F|H_e|\F_{ij}^{ab}}
=
  \tfrac{1}{4}\ol{g}_{pq}^{rs}
  \pr{\op{P}^{(a/b)}_{(i/j)}
  \g^p_i\g^q_j\h_r^a\h_s^b}
=
  \ol{g}_{ij}^{ab}
&&
  \ip{\F|H_e|\F_{ijk\cd}^{abc\cd}}
=
  0
\end{align*}
More general matrix elements $\ip{\F_I^A|H_e|\F_J^B}$ can be evaluated similarly by determining the complete cross-contractions of $\tl{a}_A^I\tl{a}_q^p\tl{a}_J^B$ and $\tl{a}_A^I\tl{a}_{rs}^{pq}\tl{a}_J^B$.
\end{rmk}

\begin{dfn}
\bmit{$\F$-normal electronic Hamiltonian.}
The \textit{$\F$-normal electronic Hamiltonian}, here denoted $H_c$, is defined as $H_c\equiv H_e-\ip{\F|H_e|\F}$.
Using \Cref{phi-normal-expansion-of-hamiltonian}, it can be expressed as follows.
\begin{align*}
&&
  H_c
=
  f_p^q
  \tl{a}_q^p
+
  \tfrac{1}{4}
  \ol{g}_{pq}^{rs}
  \tl{a}_{rs}^{pq}
\end{align*}
$H_c$ shares the same eigenstates as $H_e$, but it has a vanishing $\F$ expectation value and its eigenvalues are correlation energies $E_c=E_e-\ip{\F|H_e|\F}$ with respect to $\F$.
\end{dfn}

\begin{rmk}
\bmit{Example: Derivation of CI singles equations in KM notation.}
The eigenvalue-shifted CI singles matrix $\ip{\F_i^a|H_e|\F_j^b}-\ip{\F|H_e|\F}\d_{ij}\d^{ab}$ is equivalent to $\ip{\F_i^a|H_c|\F_j^b}$.
Its matrix elements can be determined from
\begin{align*}
&
  \ip{\F|\tl{a}_a^i\tl{a}_q^p\tl{a}_j^b|\F}
=
  \GNO{a_{a^\hole}^{i^\ptcl}a_{q^{\hole\hole}}^{p^\hole}a_{j^\ptcl}^{b^{\hole\hole}}}
+
  \GNO{a_{a^\hole}^{i^{\ptcl}}a_{q^{\ptcl}}^{p^{\ptcl\ptcl}}a_{j^{\ptcl\ptcl}}^{b^{\hole}}}
=
  \g^i_j\h_a^p\h_q^b
-
  \g_q^i\g_j^p\h_a^b
\\
&
  \ip{\F|\tl{a}_a^i\tl{a}_{rs}^{pq}\tl{a}_j^b|\F}
=
  \GNO{
    \tl{a}_{a^\hole}^{i^\ptcl}
    \tl{a}_{r^{\ptcl}s^{\hole\hole}}^{p^{\hole}q^{\ptcl\ptcl}}
    \tl{a}_{j^{\ptcl\ptcl}}^{b^{\hole\hole}}
  }
+
  \GNO{
    \tl{a}_{a^\hole}^{i^\ptcl}
    \tl{a}_{r^{\hole\hole}s^{\ptcl}}^{p^{\hole}q^{\ptcl\ptcl}}
    \tl{a}_{j^{\ptcl\ptcl}}^{b^{\hole\hole}}
  }
+
  \GNO{
    \tl{a}_{a^\hole}^{i^\ptcl}
    \tl{a}_{r^{\ptcl}s^{\hole\hole}}^{p^{\ptcl\ptcl}q^{\hole}}
    \tl{a}_{j^{\ptcl\ptcl}}^{b^{\hole\hole}}
  }
+
  \GNO{
    \tl{a}_{a^\hole}^{i^\ptcl}
    \tl{a}_{r^{\hole\hole}s^{\ptcl}}^{p^{\ptcl\ptcl}q^{\hole}}
    \tl{a}_{j^{\ptcl\ptcl}}^{b^{\hole\hole}}
  }
=
  \op{P}^{(p/q)}_{(r/s)}
  \g^i_r\h^p_a\g^q_j\h^b_s
\end{align*}
which gives
\begin{align*}
  \ip{\F_i^a|H_c|\F_j^b}
=
  f_p^q
  \pr{
    \g^i_j\h_a^p\h_q^b
  -
    \g_q^i\g_j^p\h_a^b
  }
+
  \tfrac{1}{4}
  \ol{g}_{pq}^{rs}
  \pr{
    \op{P}^{(p/q)}_{(r/s)}
    \g^i_r\h^p_a\g^q_j\h^b_s
  }
=
  f_a^b\g_j^i
-
  f_j^i\h_a^b
+
  \ol{g}_{aj}^{ib}\ .
\end{align*}
Noting that $\g_j^i=\d_j^i$ and $\h_a^b=\d_a^b$ (see \Cref{phi-opdm-ohdm}), this can be simplified to
$
  \ip{\F_i^a|H_c|\F_j^b}
=
  f_a^b\d_j^i
-
  f_j^i\d_a^b
+
  \ol{g}_{aj}^{ib}
$.
\end{rmk}


\appendix

\section{Derivation of the pull-through relation}

\begin{pro}
\label{pull-through}
\bmit{Pull-through relation.}
\textit{For arbitrary operators $x_1,\ld,x_n$ and $y$,
\begin{align*}
&&
  x_1\etc x_ny
=
  (\mp)^n
  yx_1\etc x_n
+
  \sum_{k=1}^n(\mp)^{n-k}x_1\etc[x_k,y]_{\pm}\etc x_n
\end{align*}
where $[x,y]_{\pm}\equiv xy\pm yx$.}
\\\hangindent1em\hangafter5
Proof: 
The $n=1$ case follows from the definition of the commutator brackets: $xy =\mp yx + [x,y]_{\pm}$.
Now, assume it holds for $n$ and consider $n+1$.
Since $x_1\etc x_{n+1}y=x_1\etc x_n(\mp yx_{n+1}+[x_{n+1},y]_{\pm})$, we find
\begin{align*}
  x_1\etc x_{n+1}y
=&\
\mp
  \pr{
    (\mp)^n
    yx_1\etc x_n
  +
    \sum_{k=1}^n
    (\mp)^{n-k}
    x_1\etc [x_k,y]_{\pm}\etc x_n
  }
  x_{n+1}
+
  x_1\etc x_n[x_{n+1},y]_{\pm}
\\=&\
  (\mp)^{n+1}
  yx_1\etc x_n
+
  \sum_{k=1}^{n+1}
  (\mp)^{n+1-k}
  x_1\etc[x_k,y]_{\pm}\etc x_{n+1}
\end{align*}
and, by induction, the result holds for all $n$.
\end{pro}



\end{document}
