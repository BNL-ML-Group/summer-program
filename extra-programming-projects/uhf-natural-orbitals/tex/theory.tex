\documentclass[11pt,fleqn]{article}
\usepackage[cm]{fullpage}

\usepackage{avcgreek}
\usepackage{avcfonts}
\usepackage{avcmath}
\usepackage{qcmacros}

\title{Extra Programming Project Theory\\
\textit{UHF Natural Orbitals}}
\date{}
\author{}

\begin{document}
\maketitle


The position space representation of the electron density operator is given by
\begin{align*}
&&
  \op{\rh}(\bo{r})
=
  \sum_{i=1}^{n_e}
  \d(\bo{r}-\bo{r}_i)
\end{align*}
which adds up the probability for each electron to inhabit position $\bo{r}$.
The density of a single determinant is
\begin{align*}
  \rh(\bo{r})
=
  \ip{\F|\op{\rh}(\bo{r})|\F}
=
  n\ip{\F|\d(\bo{r}-\bo{r}_1)|\F}
=&\
  \sum_{i=1}^{n_e}
  \ip{\y_i(\bo{r}_1)|\d(\bo{r}-\bo{r}_1)|\y_i(\bo{r}_1)}_{\bo{r}_1}
\\=&\
  \sum_{i_\a=1}^{n_\a}
  \f_{i_\a}^{\a*}(\bo{r})
  \f_{i_\a}^\a(\bo{r})
+
  \sum_{i_\b=1}^{n_\b}
  \f_{i_\b}^{\b*}(\bo{r})
  \f_{i_\b}^\b(\bo{r})
\end{align*}
which can be identified as the trace of the spatial one-particle density matrix.
\begin{align*}
  \underset{\bo{r}'\rightarrow\bo{r}}{\tr}
  d(\bo{r},\bo{r}')
=
  \int
  d^3\bo{r}'\,
  \d(\bo{r}-\bo{r}')d(\bo{r},\bo{r}')
=
  \rh(\bo{r})
&&
  d(\bo{r},\bo{r}')
\equiv
  \sum_{i_\a=1}^{n_\a}
  \f_{i_\a}^{\a*}(\bo{r}')
  \f_{i_\a}^\a(\bo{r})
+
  \sum_{i_\b=1}^{n_\b}
  \f_{i_\b}^{\b*}(\bo{r}')
  \f_{i_\b}^\b(\bo{r})
\end{align*}
From the position-space representation of the operator $d(\bo{r},\bo{r}')=\ip{\bo{r}|\op{d}|\bo{r}'}$, we can back out an expression for the abstract density operator $\op{d}$ as follows.
\begin{align*}
&
  d(\bo{r},\bo{r}')
=
  \sum_{i_\a=1}^{n_\a}
  \f_{i_\a}^{\a*}(\bo{r}')
  \f_{i_\a}^\a(\bo{r})
+
  \sum_{i_\b=1}^{n_\b}
  \f_{i_\b}^{\b*}(\bo{r}')
  \f_{i_\b}^\b(\bo{r})
=
  \br{\bo{r}}
  \pr{
    \sum_{i_\a=1}^{n_\a}
    \kt{\f_{i_\a}^\a}\br{\f_{i_\a}^\a}
  +
    \sum_{i_\b=1}^{n_\b}
    \kt{\f_{i_\b}^\b}\br{\f_{i_\b}^\b}
  }
  \kt{\bo{r}'}
\\&
\implies
  \op{d}
=
  \sum_{i_\a=1}^{n_\a}
  \kt{\f_{i_\a}^\a}\br{\f_{i_\a}^\a}
+
  \sum_{i_\b=1}^{n_\b}
  \kt{\f_{i_\b}^\b}\br{\f_{i_\b}^\b}
\end{align*}
This is the spatial one-particle density operator of the determinant $\F=\det(\y_1\cd \y_{n_e})$.
Expanding the molecular orbitals in terms of atomic orbital basis functions, $\f_{p_\w}^\w=\sum_\mu\x_{\mu}C_{\mu p}^\w$ where $\w\in\{\a,\b\}$, the spatial density operator of $\F$ can be expressed in terms of the Hartree-Fock density matrices $D_{\mu\nu}^\w=\sum_{i_\w=1}^{n_\w}C_{\mu i_\w}^\w C_{\nu i_\w}^\w$ as follows.
\begin{align}
\label{eq:uhf-spatial-one-particle-density-operator}
  \op{d}
=
  \sum_{i_\a=1}^{n_\a}
  \sum_{\mu\nu}
  C_{\mu i_\a}^{\a*} C_{\nu i_\a}^\a
  \kt{\x_\mu}\br{\x_\nu}
+
  \sum_{i_\b=1}^{n_\b}
  \sum_{\mu\nu}
  C_{\mu i_\b}^{\b*} C_{\nu i_\b}^\b
  \kt{\x_\mu}\br{\x_\nu}
=
  \sum_{\mu\nu}
  \pr{
    D^\a_{\mu\nu}
  +
    D^\b_{\mu\nu}
  }
  \kt{\x_\mu}\br{\x_\nu}
\end{align}
In general, the spatial natural orbitals of a wavefunction $\Y$ are the eigenfunctions of its spatial one-particle density operator.
\begin{align*}
&&
  \op{d}
  \kt{\f_p^{\mr{no}}}
=
  n_p^{\mr{no}}
  \kt{\f_p^{\mr{no}}}
\end{align*}
Expanding the natural orbitals in terms of AO basis functions $\f_p^{\mr{no}}=\sum_\nu \x_\nu C_{\nu p}^{\mr{no}}$, projecting by $\br{\x_\mu}$, and expanding $\op{d}$ according to eq~\ref{eq:uhf-spatial-one-particle-density-operator}, the eigenvalue equation becomes
\begin{align*}
  \sum_\nu
  \sum_{\rh\si}
  \ip{\x_\mu|\x_\rh}
  \pr{
    D^\a_{\rh\si}
  +
    D^\b_{\rh\si}
  }
  \ip{\x_\si|\x_\nu}
  C_{\nu p}^{\mr{no}}
=
  n_p^{\mr{no}}
  \sum_\nu
  \ip{\x_\mu|\x_\nu}
  C_{\nu p}^{\mr{no}}
\end{align*}
which can be expressed in matrix notation as follows.
\begin{align*}
  \bo{S}
  \bo{D}
  \bo{S}
  \bo{C}^{\mr{no}}
=
  \bo{S}
  \bo{C}^{\mr{no}}
  \tl{\bo{D}}^{\mr{no}}
&&
  \bo{D}
=
  \bo{D}^\a + \bo{D}^\b
=
  [D_{\mu\nu}^\a + D_{\mu\nu}^\b]
&&
  \bo{C}^{\mr{no}}
=
  [C_{\mu p}^{\mr{no}}]
&&
  \tl{\bo{D}}^{\mr{no}}
=
  [n_p^{\mr{no}} \d_{pq}]
\end{align*}
Multiplying both sides by the orthogonalizer $\bo{X}\equiv\bo{S}^{-\fr{1}{2}}$ reduces this equation to a standard symmetric matrix diagonalization.
\begin{align*}
  \ol{\bo{D}}\,
  \ol{\bo{C}}^{\mr{no}}
=
  \ol{\bo{C}}^{\mr{no}}
  \tl{\bo{D}}^{\mr{no}}
&&
  \ol{\bo{D}}
=
  \bo{S}^{\fr{1}{2}}\bo{D}\bo{S}^{\fr{1}{2}}
&&
  \ol{\bo{C}}^{\mr{no}}
=
  \bo{S}^{\fr{1}{2}}\bo{C}^{\mr{no}}
\end{align*}



\end{document}
