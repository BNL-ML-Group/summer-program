\documentclass[11pt]{article}

\usepackage[cm]{fullpage}

\usepackage{tikz}
\usepackage{amsmath}

\usepackage{listings}
\lstset{basicstyle=\ttfamily\small}
\lstset{literate={~} {$\sim$}{1}}
\lstset{showstringspaces=false}
\lstset{language=Python}

\newcommand{\linp}[1]{\lstinputlisting{#1}{}}
\newcommand{\linl}[1]{\lstinline{#1}{}}



\usepackage[cm]{fullpage}
\usepackage{mathtools} %includes amsmath
\usepackage{amsfonts}
\usepackage{listings}
\usepackage{lmodern}
\usepackage{scrextend}
\usepackage{enumitem}
\usepackage{graphicx}
\usepackage{hyperref}
\usepackage{tikz}
\usepackage{soul}
\usepackage{bm}

\newcommand{\bo}[1]{\ensuremath{\mathbf{#1}}}
\newcommand{\tl}[1]{\ensuremath{\tilde{#1}}}
\newcommand{\pt}{\partial}

\newcommand{\ttf}[1]{{\ttfamily #1}}
\newcommand{\s}{\textbackslash}
\renewcommand{\sp}{\ \ \ \ \ \ \ \ \ \ }
\newcommand{\fr}[2]{\dfrac{#1}{#2}}
\newcommand{\pr}[1]{\left(#1\right)}
\newcommand{\ord}[1]{\ensuremath{^{(#1)}}}
\newcommand{\ma}[1]{\left(\begin{matrix}#1\end{matrix}\right)}
\newcommand{\ar}[1]{\ensuremath{\begin{matrix} #1 \end{matrix}}}
\newcommand{\pd}[3]{\ensuremath{ \dfrac{ \partial^{#1} #2 }{\partial #3 ^{#1}}}}
\newcommand{\bigo}{\ensuremath{\mathcal{O}}}
\newcommand{\eth}{\ensuremath{^\text{th}}}
\newcommand{\ld}{\ensuremath{\ldots}}
\newcommand{\vd}{\ensuremath{\vdots}}
\newcommand{\dd}{\ensuremath{\ddots}}
\newcommand{\miniar}[1]{\ensuremath{\begin{smallmatrix}#1\end{smallmatrix}}}
\newcommand{\Nu}{\ensuremath{\mathrm{Nu}}}
\newcommand{\op}[1]{\ensuremath{\hat{#1}}}
\newcommand{\Y}{\ensuremath{\Psi}}
\newcommand{\y}{\ensuremath{\psi}}
\newcommand{\si}{\ensuremath{\sigma}}
\newcommand{\w}{\ensuremath{\omega}}
\renewcommand{\d}{\ensuremath{\delta}}
\newcommand{\D}{\ensuremath{\Delta}}
\newcommand{\la}{\ensuremath{\lambda}}
\newcommand{\La}{\ensuremath{\Lambda}}
\newcommand{\mf}[1]{\ensuremath{\mathfrak{#1}}}
\newcommand{\e}{\ensuremath{\bm\varepsilon}}



\title{Programming Project 1 Theory\\
\textit{The harmonic approximation to the nuclear Schr\"odinger equation}}
\date{}
\author{}

\begin{document}
\maketitle
\vspace{-2cm}

\subsection*{The Born-Oppenheimer Approximation}
Under the Born-Oppenheimer approximation the stationary-states
\footnote{\url{http://en.wikipedia.org/wiki/Stationary_state}} of nuclear
motion $\Y_\Nu$ arise as solutions of the nuclear motion equation.
\begin{align}
\label{pes}
    \op{H}_\Nu
    \Y_\Nu(\bo{X})
=
    E
    \Y_\Nu(\bo{X})
\sp
    \op{H}_\Nu
\equiv
    \op{T}_\Nu+E_e(\bo{X})
\end{align}
The potential energy term $E_e(\bo{X})$ in the nuclear motion Hamiltonian is
defined through the clamped-nuclei Schr\"odinger equation.
\begin{align}
\label{clamped-se}
    \op{H}_e(\bo{X})
    \Y_e(\bo{r}_1,\ld,\bo{r}_n;\bo{X})
=
    E_e(\bo{X})
    \Y_e(\bo{r}_1,\ld,\bo{r}_n;\bo{X})
\end{align}
Equation (\ref{clamped-se}) can be solved approximately at any point
$\bo{X}=(\bo{R}_1,\ld,\bo{R}_N)$ using your favorite electronic structure
software package (\textsc{Psi4}, CFOUR, Molpro, Gaussian, etc.) and electronic
structure method (MP2, CCSD, CISD, etc.).


\subsection*{Normal Coordinates and the Vibrational Schr\"odinger Equation}

One of the major barriers to solving the nuclear motion equation (\ref{pes}) is
the difficulty of determining the potential energy surface $E_e(\bo{X})$ at a
sufficiently large number of points. Since the nuclei generally remain
relatively localized about an equilibrium configuration $\bo{X}_0$, a good
first approach is to approximate the potential surface by a Taylor expansion.
\begin{align*}
    E_e(\bo{X})
\approx
    E_e(\bo{X}_0)
+\sum_A^{3N}
    \pr{\pd{}{E_e}{X_A}}_0
    \D X_A
+\fr{1}{2}\sum_{AB}^{3N}
    \pr{\fr{\pt^2E_e}{\pt X_A\pt X_B}}_0
    \D X_A
    \D X_B
    \sp \D\bo{X}=\bo{X}-\bo{X}_0
\end{align*}
Equilibrium structures occur at minima on the potential energy surface, so that
the second term is zero and only the quadratic term remains.
\begin{align}
    E_e(\bo{X})
\approx
    E_e(\bo{X}_0)
+\fr{1}{2}\sum_{AB}^{3N}
    \pr{\fr{\pt^2E_e}{\pt X_A\pt X_B}}_0
    \D X_A
    \D X_B
\end{align}
For convenience, we use mass-weighted coordinates $\tl{X}_A \equiv \sqrt{M_A}\D
X_A$ in order to remove the explicit dependence of the kinetic energy operator
on nuclear masses. \footnote{Explicitly, we can write
$\fr{\op{\bo{P}}_A^2}{2M_A} = -\fr{1}{2M_A}
\pd{2}{}{X_A}=-\fr{1}{2}\pd{2}{}{(\sqrt{M_A}\D
X_A)}=-\fr{1}{2}\pd{2}{}{\tl{X}_A}$. ($\hbar=1$ since we are working in atomic
units)}
\begin{align}
    \op{T}_\Nu
=
    -\fr{1}{2}\sum_A^{3N} \pd{2}{}{\tl{X}_A}
\end{align}
The nuclear motion equation (\ref{pes}) then becomes
\begin{align}
\label{pes-intermediate}
\pr{
-\fr{1}{2}\sum_A^{3N}
    \pd{2}{}{\tl{X}_A}
+\fr{1}{2}\sum_{AB}^{3N}
    \pr{\fr{\pt^2E_e}{\pt \tl{X}_A\pt \tl{X}_B}}_0
    \tl{X}_A
    \tl{X}_B
}
    \Y_\Nu(\tl{\bo{X}})
=
    E_\Nu
    \Y_\Nu(\tl{\bo{X}})
\sp
    E_\Nu
\equiv
    E - E_e(\bo{X}_0)
\end{align}
which brings us to the crucial step that motivates this project.

\paragraph{the crucial step that motivates this project:} Note that, if the
mass-weighted Hessian matrix
\begin{align}
    (\tl{\bo{H}}_0)_{AB}
=
    \pr{\fr{\pt^2 E_e}{\pt\tl{X}_A\pt\tl{X}_B}}_0
\end{align}
were diagonal, we would have a sum of harmonic oscillator Hamiltonians
\footnote{The Hamiltonian of a single harmonic oscillator is
$\op{H}=\fr{\op{p}^2}{2m}+\fr{m\w^2\op{x}^2}{2}$, or
$\op{H}=-\fr{1}{2}\pd{2}{}{q}+\fr{\w^2}{2}q^2$ in mass-weighted coordinates
with atomic units.  The frequency of oscillation is $\w$.} in equation
(\ref{pes-intermediate}). This dream of a simple Hamiltonian can be realized by
choosing a new set of coordinates $\{q_1,\ld,q_{3N}\}$, called {\it normal coordinates}, in terms of which the Hessian is diagonal.
Since the mass-weighted Hessian $\tl{\bo{H}}_0$ is real and symmetric, it can
be diagonalized by a real orthogonal matrix $\bo{\tl{Q}}$
\footnote{\url{http://en.wikipedia.org/wiki/Symmetric_matrix\#Decomposition}}
\begin{align}
    \tl{\bo{H}}_0
=
    \bo{\tl{Q}}\bo{K}\bo{\tl{Q}}^T
\sp
    \bo{K}
=
    \ma{ k_1   &   0   &  \ld  &   0      \\
           0   & k_2   &  \ld  &   0      \\
          \vd  &  \vd  &  \dd  &  \vd     \\
           0   &   0   &  \ld  & k_{3N}   }
\sp
    \bo{\tl{Q}}\bo{\tl{Q}}^T=\bo{\tl{Q}}^T\bo{\tl{Q}}=\bo{1}
\end{align}
where the columns of $\bo{\tl{Q}}$ are eigenvectors $\bm{\tl{q}}_1,\ld,\bm{\tl{q}}_{3N}$ of
$\tl{\bo{H}}_0$.
Back-transforming to un-mass-weighted Cartesian space
gives the \textit{normal modes} of the system.
\begin{align}
&
  \bm{q}_1
=
  \bo{M}^{-1/2}
  \bm{\tl{q}}_1
&&
  \bm{q}_2
=
  \bo{M}^{-1/2}
  \bm{\tl{q}}_2
&&
  \cdots
&&
  \bm{q}_{3N}
=
  \bo{M}^{-1/2}
  \bm{\tl{q}}_{3N}
&&
  \text{where  }
    \bo{M}
\equiv
    \ma{ m_1 & 0   & 0   & 0   & \cdots\\
         0   & m_1 & 0   & 0   & \cdots\\
         0   & 0   & m_1 & 0   & \cdots\\
         0   & 0   & 0   & m_2 & \cdots\\
         \vd & \vd & \vd & \vd & \dd}   
\end{align}
Each \textit{normal coordinate} $q$ represents a motion along the \textit{normal mode} $\bm{q}_A$.


Expressing equation (\ref{pes-intermediate}) in terms of these newfangled
coordinates, we are left with 
\begin{align}
\label{vib-se}
\pr{
-\fr{1}{2}\sum_A^{3N}
    \pd{2}{}{q_A}
+\fr{1}{2}\sum_A^{3N}
    k_A q_A^2
}
    \Y_\Nu(\bo{q})
=
    E_\Nu
    \Y_\Nu(\bo{q})
\end{align}
(details of the coordinate transformation are shown in an appendix). The
Hamiltonian in (\ref{vib-se}) has the marvelously simple form we were hoping
for:
\begin{align}
\label{harm-hamiltonian}
    \op{H}_\Nu
\approx
    \sum_{A=1}^{3N}
    \op{\mf{h}}_A
\sp
    \op{\mf{h}}_A
\equiv
-\fr{1}{2}
    \pd{2}{}{q_A}
+\fr{1}{2}
    k_A q_A^2
\end{align}
The only approximation for $\op{H}_\Nu$ here is the truncation of the potential
energy surface Taylor expansion at second order. As long as $k_A$ is a
positive real number, $\op{\mf{h}}_A$ is a harmonic oscillator Hamiltonian and
we can make the identification $k_A=\w_A^2$ where $\w_A$ is the oscillator
frequency in a.u. It will turn out that, at an equilibrium geometry,
$\tl{\bo{H}}_0$ has several zero eigenvalues -- one for each translational and
rotational degree of freedom for the entire nuclear framework with respect to
its center of mass. For these motions, $\op{\mf{h}}_A$ actually takes the form
of a free-particle Hamiltonian
\footnote{\url{http://en.wikipedia.org/wiki/Free_particle\#Non-Relativistic_Quantum_Free_Particle}}
\begin{align}
    \op{\mf{h}}_A
    = -\fr{1}{2} \pd{2}{}{q_A}
\end{align}
although this form is somewhat misleading for rotational motion. In order to
get a clearer picture of what a normal coordinate physically represents, it
helps to shift to a classical picture for a moment.



\subsubsection*{Solutions for Nuclear Motion in a Quadratic Well}
Here I will simply report the quantum mechanical solution to the translational
and vibrational components of equation (\ref{vib-se}).
\begin{align*}
\pr{\sum_{A=1}^{3N}
    \op{\mf{h}}_A }
    \Y_\Nu(\bo{q})
=
    E_\Nu
    \Y_\Nu(\bo{q})
\sp
    \op{\mf{h}}_A
=
-\fr{1}{2}
    \pd{2}{}{q_A}
+\fr{1}{2}
    k_A q_A^2
\end{align*}
Since the Hamiltonian separates into a sum of independent Hamiltonians each
involving only one coordinate, the form of the wavefunction is
\begin{align}
    \Y_\Nu(q_1,\ld,q_{3N})
=
    \y_1(q_1)
    \y_2(q_2)
    \cdots
    \y_{3N}(q_{3N})
\end{align}
where each one-coordinate wavefunction $\y_A(q_A)$ is an eigenfunction (or,
rather, {\it one of the} eigenfunctions) of the corresponding operator,
$\op{\mf{h}}_A$.
\begin{align}
    \op{\mf{h}}_A
    \y_A(q_A)
=
    \e_A
    \y_A(q_A)
\end{align}
We will only need solutions to the 1D free-particle and 1D harmonic oscillator
Schr\"odinger equations for the present discussion, which can be found in any
introductory quantum mechanics text worth reading.\footnote{e.g. R.\ Shankar
{\it Principles of Quantum Mechanics}, which contains detailed discussions of
both the free particle and harmonic oscillator. An electronic copy can be found
\href{http://home.basu.ac.ir/\~psu/Books/\%5BRamamurti_Shankar\%5D_Principles_of_Quantum_Mechanic\%28BookFi.org\%29.pdf}{here}.}

For the three normal coordinates that correspond to translations of the
molecule as a whole, the potential term in $\op{\mf{h}}$ vanishes (since
$k=0$) and we only need to solve a free-particle-like problem.
\begin{align}
-\fr{1}{2}
    \pd{2}{\y(q)}{q}
=
    \e
    \y(q)
\end{align}
Solutions to this differential equation are given by
\begin{align}
    \y_{P}(q)
=
    e^{iP(q-q_0)}
\end{align}
where $q_0$ is some undetermined phase shift. The distribution of
eigenvalues is continuous in this case, $\e_{P}=\fr{P^2}{2}$, where $P$
corresponds to a component of the linear momentum of the molecule as a whole.

Although $k=0$ also holds for rotational modes at equilibrium, normal
coordinates are only appropriate for parametrizing infinitesimal rotations and
quickly break down for larger rotational motions.
Furthermore, unlike translations, these modes are not truly
``external'' since they change the relative positions of the nuclei (and
therefore change the energy for finite displacements). Proper parametrization
of a full rotation about the center of mass in terms of fixed normal
coordinates would actually require linear combinations of rotational modes
 with vibrational ones, and these
motions are in general coupled to each other. For our purposes, we will simply
note one can, to a good first approximation, ignore the interference of
rotations when considering vibrational motion.

The remaining $3N-6$ coordinates ($3N-5$ for linear molecules) will always have
$k>0$ at an equilibrium geometry and so lead to true harmonic oscillator
motion. The solutions to this problem
\begin{align}
-\fr{1}{2}
    \pd{2}{\y(q)}{q}
+\fr{1}{2}
    k
    q^2
=
    \e
    \y(q)
\end{align}
are given by
\begin{align}
    \y_{n}(q)
=
\fr{1}{\sqrt{2^n n!}}\pr{\fr{\w}{\pi}}^{\frac{1}{4}}
    e^{-\frac{1}{2}\w q^2}
    \ \text{He}_n(\sqrt{\w}q)
\sp
    \w^2=k
\end{align}
with energies $\e_{n}=\w(n+\frac{1}{2})$, where $\text{He}_n(x)$ is the
$n$\eth\ Hermite polynomial.
\begin{align}
    \text{He}_n(x)
=   (-1)^n
    e^{x^2}
    \pr{\fr{d^n}{dx^n}e^{-x^2}}
\end{align}
Hence, the solution to the {\it vibrational Schr\"odinger equation}
\footnote{This is simply equation (\ref{vib-se}) with translational and
rotational modes omitted.}
\begin{align}
\sum_{A=1}^{3N-6}
    \pr{-\fr{1}{2}\pd{2}{}{q_A}+\w_A^2q_A^2}
    \Y_\text{vib}(\bo{q})
=
    E_\text{vib}
    \Y_\text{vib}(\bo{q})
\end{align}
is given by
\begin{align}
    \Y_{\bo{n}}(\bo{q})
    = \prod_{A=1}^{3N-6}
    \y_{n_A}(q_A)
\end{align}
with vibrational energy
\begin{align}
    E_{\bo{n}}
=
\sum_{A=1}^{3N-6}
    \w_A
    \pr{n_A+\fr{1}{2}}
\end{align}
The total molecular energy at equilibrium is the sum of translational,
rotational, vibrational, and electronic contributions.
\begin{align}
    E
    = \fr{\bo{P}_\text{CM}^2}{2M_\text{tot}}
    + E_\text{vib}
    + E_\text{rot}
    + E_e(\bo{X}_0)
\end{align}




\end{document}