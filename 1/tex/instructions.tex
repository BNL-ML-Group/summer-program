\documentclass[fleqn]{article}

\usepackage{listings}
\lstset{basicstyle=\ttfamily\small}
\lstset{literate={~} {$\sim$}{1}}
\lstset{showstringspaces=false}
\lstset{language=Python}
\usepackage{scrextend}
\newcommand{\linp}[1]{\lstinputlisting{#1}{}}
\newcommand{\linl}[1]{\lstinline{#1}{}}

\usepackage[cm]{fullpage}
\usepackage{mathtools} %includes amsmath
\usepackage{amsfonts}
\usepackage{bm}
\usepackage{url}
%greek letters
\renewcommand{\a}{\alpha}    %alpha
\renewcommand{\b}{\beta}     %beta
\newcommand{\g}{\gamma}      %gamma
\newcommand{\G}{\Gamma}      %Gamma
\renewcommand{\d}{\delta}    %delta
\newcommand{\D}{\Delta}      %Delta
\newcommand{\e}{\varepsilon} %epsilon
\newcommand{\ev}{\epsilon}   %epsilon*
\newcommand{\z}{\zeta}       %zeta
\newcommand{\h}{\eta}        %eta
\renewcommand{\th}{\theta}   %theta
\newcommand{\Th}{\Theta}     %Theta
\newcommand{\io}{\iota}      %iota
\renewcommand{\k}{\kappa}    %kappa
\newcommand{\la}{\lambda}    %lambda
\newcommand{\La}{\Lambda}    %Lambda
\newcommand{\m}{\mu}         %mu
\newcommand{\n}{\nu}         %nu %xi %Xi %pi %Pi
\newcommand{\p}{\rho}        %rho
\newcommand{\si}{\sigma}     %sigma
\newcommand{\siv}{\varsigma} %sigma*
\newcommand{\Si}{\Sigma}     %Sigma
\renewcommand{\t}{\tau}      %tau
\newcommand{\up}{\upsilon}   %upsilon
\newcommand{\f}{\phi}        %phi
\newcommand{\F}{\Phi}        %Phi
\newcommand{\x}{\chi}        %chi
\newcommand{\y}{\psi}        %psi
\newcommand{\Y}{\Psi}        %Psi
\newcommand{\w}{\omega}      %omega
\newcommand{\W}{\Omega}      %Omega
%ornaments
\newcommand{\eth}{\ensuremath{^\text{th}}}
\newcommand{\rst}{\ensuremath{^\text{st}}}
\newcommand{\ond}{\ensuremath{^\text{nd}}}
\newcommand{\ord}[1]{\ensuremath{^{(#1)}}}
\newcommand{\dg}{\ensuremath{^\dagger}}
\newcommand{\bigo}{\ensuremath{\mathcal{O}}}
\newcommand{\tl}{\ensuremath{\tilde}}
\newcommand{\ol}[1]{\ensuremath{\overline{#1}}}
\newcommand{\ul}[1]{\underline{#1}}
\newcommand{\op}[1]{\ensuremath{\hat{#1}}}
\newcommand{\ot}{\ensuremath{\otimes}}
\newcommand{\wg}{\ensuremath{\wedge}}
%text
\newcommand{\tr}{\ensuremath{\hspace{1pt}\mathrm{tr}\hspace{1pt}}}
\newcommand{\Alt}{\ensuremath{\mathrm{Alt}}}
\newcommand{\sgn}{\ensuremath{\mathrm{sgn}}}
\newcommand{\occ}{\ensuremath{\mathrm{occ}}}
\newcommand{\vir}{\ensuremath{\mathrm{vir}}}
\newcommand{\spn}{\ensuremath{\mathrm{span}}}
\newcommand{\vac}{\ensuremath{\mathrm{vac}}}
\newcommand{\bs}{\ensuremath{\text{\textbackslash}}}
\newcommand{\im}{\ensuremath{\mathrm{im}\hspace{1pt}}}
\renewcommand{\sp}{\hspace{30pt}}
%dots
\newcommand{\ld}{\ensuremath{\ldots}}
\newcommand{\cd}{\ensuremath{\cdots}}
\newcommand{\vd}{\ensuremath{\vdots}}
\newcommand{\dd}{\ensuremath{\ddots}}
\newcommand{\etc}{\ensuremath{\mathinner{\mkern-1mu\cdotp\mkern-2mu\cdotp\mkern-2mu\cdotp\mkern-1mu}}}
%fonts
\newcommand{\bmit}[1]{{\bfseries\itshape\mathversion{bold}#1}}
\newcommand{\mc}[1]{\ensuremath{\mathcal{#1}}}
\newcommand{\mb}[1]{\ensuremath{\mathbb{#1}}}
\newcommand{\mf}[1]{\ensuremath{\mathfrak{#1}}}
\newcommand{\mr}[1]{\ensuremath{\mathrm{#1}}}
\newcommand{\bo}[1]{\ensuremath{\mathbf{#1}}}
%styles
\newcommand{\ts}{\textstyle}
\newcommand{\ds}{\displaystyle}
\newcommand{\phsub}{\ensuremath{_{\phantom{p}}}}
\newcommand{\phsup}{\ensuremath{^{\phantom{p}}}}
%fractions, derivatives, parentheses, brackets, etc.
\newcommand{\pr}[1]{\ensuremath{\left(#1\right)}}
\newcommand{\brk}[1]{\ensuremath{\left[#1\right]}}
\newcommand{\fr}[2]{\ensuremath{\dfrac{#1}{#2}}}
\newcommand{\pd}[2]{\ensuremath{\frac{\partial#1}{\partial#2}}}
\newcommand{\fd}[2]{\ensuremath{\frac{d #1}{d #2}}}
\newcommand{\pt}{\ensuremath{\partial}}
\newcommand{\br}[1]{\ensuremath{\langle#1|}}
\newcommand{\kt}[1]{\ensuremath{|#1\rangle}}
\newcommand{\ip}[1]{\ensuremath{\langle#1\rangle}}
%structures
\newcommand{\eqn}[1]{(\ref{#1})}
\newcommand{\ma}[1]{\ensuremath{\begin{bmatrix}#1\end{bmatrix}}}
\newcommand{\ar}[1]{\ensuremath{\begin{matrix}#1\end{matrix}}}
\newcommand{\miniar}[1]{\ensuremath{\begin{smallmatrix}#1\end{smallmatrix}}}
\newcommand{\rad}{\ensuremath{\mr{rad}}}


\title{Programming Project 1: Frequencies\\
\textit{Computing frequencies and normal modes from the Hessian matrix}}
\author{}
\date{}

\begin{document}

\maketitle
\vspace{-1cm}
\noindent
Write a script to read in a Hessian matrix, compute the corresponding frequencies and normal modes, and print them to a file in Jmol's \linl{.xyz} format.
You should import your \linl{Molecule} class from Project 0 and make use of it in this script.

\subsection*{Extra Files}
\begin{center}
\begin{tabular}{p{0.25\textwidth}@{}p{0.75\textwidth}}
  \ul{file name} & \ul{description} \\
  \linl{molecule.xyz}
  & sample \linl{.xyz} file for H$_2$O \\ 
  \linl{hessian.dat}
  & sample Hessian matrix for H$_2$O \\
  \linl{project1_answer.xyz}
  & normal modes and frequencies for H$_2$O
\end{tabular}
\end{center}


\subsection*{Equations}
Let $N$ be the number of atoms and let $m_A$ and $(x_A, y_A, z_A)$ be the mass and Cartesian coordinates of the $A$\eth\ atom.
\begin{align}
\label{hessian-dfn}
  (\bo{H})_{AB}
=&\
  \pd{^2E}{X_A\pt X_B}
&
  (X_{3A-2}, X_{3A-1}, X_{3A-0})=&\ (x_A, y_A, z_A)
&&
  \text{for $A\in\{1,\ld,N\}$}
\\[5pt]
\label{mwhessian-dfn}
  (\bo{\tl{H}})_{AB}
=&\
  \fr{(\bo{H})_{AB}}{\sqrt{M_AM_B}}
&
  (M_{3A-2}, M_{3A-1}, M_{3A-0})=&\ (m_A, m_A, m_A)
&&
  \text{for $A\in\{1,\ld,N\}$}
\\[5pt]
\label{mwhessian-eigh}
  \bo{\tl{H}}\hspace{2pt}\bm{\tl{q}}_A
=&\
  k_A\bm{\tl{q}}_A
&&&&
  \text{for $A\in\{1,\ld,N\}$}
\\[5pt]
\label{backtransform-eigvecs}
  (\bm{q}_A)_B
=&\
  \fr{(\bm{\tl{q}}_A)_B}{\sqrt{M_B}}
&&&&
  \text{for $A, B\in\{1,\ld,N\}$}
\end{align}
\begin{align}
\label{conversion}
&
  k_A \left[\fr{E_h\rad^2}{a_0^2u}\right]
=
  k_A \left(\fr{E_h[\mr{J}]}{a_0[\mr{m}]^2u[kg]}\right)
  \left[\fr{\rad^2}{\mr{s}^2}\right]
&&
  \nu_A \left[\mr{Hz}\right]
=
  \fr{1}{2\pi}\left[\fr{\mr{cyc}}{\rad}\right]
  \cdot
  \sqrt{k_A \left[\fr{\rad^2}{\mr{s}^2}\right]}
&&
  \tl{\nu}_A \left[\mr{cm}^{-1}\right]
=
  \fr{\nu_A\left[\mr{Hz}\right]}{c\ [\mr{cm}/\mr{s}]}
\end{align}



\subsection*{Procedure}

Steps 3.--7. should be implemented in a separate \linl{frequencies()} function that takes a \linl{Molecule} object and a \linl{numpy.array} Hessian matrix as arguments.
\begin{enumerate}
  \item read in the Hessian matrix (defined in equation \ref{hessian-dfn}) from \linl{hessian.dat}
  \item build a \linl{Molecule} object from \linl{molecule.xyz}
  \item build the mass-weighted Hessian matrix (equation \ref{mwhessian-dfn})
  \item compute the eigenvalues ($k_A$) and eigenvectors ($\bm{\tl{q}}_A$) of the mass-weighted Hessian matrix (equation \ref{mwhessian-eigh})
  \item un-mass-weight the eigenvectors (equation \ref{backtransform-eigvecs}) to get normal coordinates
  \item determine the spatial frequencies $\tl{\n}_A$ in $\mr{cm}^{-1}$ from your force constants $k_A$, which are in atomic units (equation \ref{conversion})
  \item write the normal modes to a file in \linl{.xyz} format, with frequencies neatly displayed in the comment lines
\end{enumerate}
After you are finished, check your answers against \linl{project1_answer.xyz} and then try visualizing the normal modes in Jmol.

\newpage
\subsubsection*{Description of \linl{.xyz} format for normal modes}
The file format for visualizing a single normal mode in Jmol is
\begin{addmargin}{5cm}{}
\begin{lstlisting}[language=c++]
N
COMMENT LINE 1
A1 x1 y1 z1   dx1 dy1 dz1
A2 x2 y2 z2   dx2 dy2 dz2
...                        
AN xN yN zN   dxN dyN dzN
\end{lstlisting}
\end{addmargin}
which amounts to a standard \linl{.xyz} geometry file (distance units \AA) with the displacement vector for each atom printed next to
its Cartesian coordinates. For visualizing multiple motions, this format is
simply repeated with an intervening empty line.
\begin{addmargin}{5cm}{}
\begin{lstlisting}[language=c++]
N
COMMENT LINE 1
A1 x1 y1 z1   dx11 dy11 dz11
A2 x2 y2 z2   dx12 dy12 dz12
...                        
AN xN yN zN   dx1N dy1N dz1N
                           
N                          
COMMENT LINE 2             
A1 x1 y1 z1   dx21 dy21 dz21
A2 x2 y2 z2   dx22 dy22 dz22
...                        
AN xN yN zN   dx2N dy2N dz2N
                           
...                        
                           
N                          
COMMENT LINE N             
A1 x1 y1 z1   dxN1 dyN1 dzN1
A2 x2 y2 z2   dxN2 dyN2 dzN2
...                        
AN xN yN zN   dxNN dyNN dzNN
\end{lstlisting}
\end{addmargin}

\end{document}