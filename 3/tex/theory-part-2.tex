\documentclass[11pt,fleqn]{article}
\usepackage[cm]{fullpage}
\usepackage{mathtools} %includes amsmath
\usepackage{amsfonts}
\usepackage{bm}
\usepackage{url}
%greek letters
\renewcommand{\a}{\alpha}    %alpha
\renewcommand{\b}{\beta}     %beta
\newcommand{\g}{\gamma}      %gamma
\newcommand{\G}{\Gamma}      %Gamma
\renewcommand{\d}{\delta}    %delta
\newcommand{\D}{\Delta}      %Delta
\newcommand{\e}{\varepsilon} %epsilon
\newcommand{\ev}{\epsilon}   %epsilon*
\newcommand{\z}{\zeta}       %zeta
\newcommand{\h}{\eta}        %eta
\renewcommand{\th}{\theta}   %theta
\newcommand{\Th}{\Theta}     %Theta
\newcommand{\io}{\iota}      %iota
\renewcommand{\k}{\kappa}    %kappa
\newcommand{\la}{\lambda}    %lambda
\newcommand{\La}{\Lambda}    %Lambda
\newcommand{\m}{\mu}         %mu
\newcommand{\n}{\nu}         %nu %xi %Xi %pi %Pi
\newcommand{\p}{\rho}        %rho
\newcommand{\si}{\sigma}     %sigma
\newcommand{\siv}{\varsigma} %sigma*
\newcommand{\Si}{\Sigma}     %Sigma
\renewcommand{\t}{\tau}      %tau
\newcommand{\up}{\upsilon}   %upsilon
\newcommand{\f}{\phi}        %phi
\newcommand{\F}{\Phi}        %Phi
\newcommand{\x}{\chi}        %chi
\newcommand{\y}{\psi}        %psi
\newcommand{\Y}{\Psi}        %Psi
\newcommand{\w}{\omega}      %omega
\newcommand{\W}{\Omega}      %Omega
%ornaments
\newcommand{\eth}{\ensuremath{^\text{th}}}
\newcommand{\rst}{\ensuremath{^\text{st}}}
\newcommand{\ond}{\ensuremath{^\text{nd}}}
\newcommand{\ord}[1]{\ensuremath{^{(#1)}}}
\newcommand{\dg}{\ensuremath{^\dagger}}
\newcommand{\bigo}{\ensuremath{\mathcal{O}}}
\newcommand{\tl}{\ensuremath{\tilde}}
\newcommand{\ol}[1]{\ensuremath{\overline{#1}}}
\newcommand{\ul}[1]{\ensuremath{\underline{#1}}}
\newcommand{\op}[1]{\ensuremath{\hat{#1}}}
\newcommand{\ot}{\ensuremath{\otimes}}
\newcommand{\wg}{\ensuremath{\wedge}}
%text
\newcommand{\tr}{\ensuremath{\hspace{1pt}\mathrm{tr}\hspace{1pt}}}
\newcommand{\Alt}{\ensuremath{\mathrm{Alt}}}
\newcommand{\sgn}{\ensuremath{\mathrm{sgn}}}
\newcommand{\occ}{\ensuremath{\mathrm{occ}}}
\newcommand{\vir}{\ensuremath{\mathrm{vir}}}
\newcommand{\spn}{\ensuremath{\mathrm{span}}}
\newcommand{\vac}{\ensuremath{\mathrm{vac}}}
\newcommand{\bs}{\ensuremath{\text{\textbackslash}}}
\newcommand{\im}{\ensuremath{\mathrm{im}\hspace{1pt}}}
\renewcommand{\sp}{\hspace{30pt}}
%dots
\newcommand{\ld}{\ensuremath{\ldots}}
\newcommand{\cd}{\ensuremath{\cdots}}
\newcommand{\vd}{\ensuremath{\vdots}}
\newcommand{\dd}{\ensuremath{\ddots}}
\newcommand{\etc}{\ensuremath{\mathinner{\mkern-1mu\cdotp\mkern-2mu\cdotp\mkern-2mu\cdotp\mkern-1mu}}}
%fonts
\newcommand{\bmit}[1]{{\bfseries\itshape\mathversion{bold}#1}}
\newcommand{\mc}[1]{\ensuremath{\mathcal{#1}}}
\newcommand{\mb}[1]{\ensuremath{\mathbb{#1}}}
\newcommand{\mf}[1]{\ensuremath{\mathfrak{#1}}}
\newcommand{\mr}[1]{\ensuremath{\mathrm{#1}}}
\newcommand{\bo}[1]{\ensuremath{\mathbf{#1}}}
%styles
\newcommand{\ts}{\textstyle}
\newcommand{\ds}{\displaystyle}
\newcommand{\phsub}{\ensuremath{_{\phantom{p}}}}
\newcommand{\phsup}{\ensuremath{^{\phantom{p}}}}
%fractions, derivatives, parentheses, brackets, etc.
\newcommand{\pr}[1]{\ensuremath{\left(#1\right)}}
\newcommand{\brk}[1]{\ensuremath{\left[#1\right]}}
\newcommand{\fr}[2]{\ensuremath{\dfrac{#1}{#2}}}
\newcommand{\pd}[2]{\ensuremath{\frac{\partial#1}{\partial#2}}}
\newcommand{\pt}{\ensuremath{\partial}}
\newcommand{\br}[1]{\ensuremath{\langle#1|}}
\newcommand{\kt}[1]{\ensuremath{|#1\rangle}}
\newcommand{\ip}[1]{\ensuremath{\langle#1\rangle}}
\newcommand{\NO}[1]{\ensuremath{{\bm{:}}#1{}{\bm{:}}}}
\newcommand{\cmtr}[2]{\ensuremath{[\cdot,#2]^{#1}}}
\newcommand{\cmtl}[2]{\ensuremath{[#2,\cdot]^{#1}}}
%structures
\newcommand{\eqn}[1]{(\ref{#1})}
\newcommand{\ma}[1]{\ensuremath{\begin{bmatrix}#1\end{bmatrix}}}
\newcommand{\ar}[1]{\ensuremath{\begin{matrix}#1\end{matrix}}}
\newcommand{\miniar}[1]{\ensuremath{\begin{smallmatrix}#1\end{smallmatrix}}}
%contractions
\usepackage{simplewick}
\usepackage[nomessages]{fp}
\newcommand{\ctr}[6][0]{\FPeval\height{1.0+#1*0.5}\ensuremath{\contraction[\height ex]{#2}{#3}{#4}{#5}}}
%math sections
\usepackage{amsthm}
\usepackage{thmtools}
\declaretheoremstyle[spaceabove=10pt,spacebelow=10pt,bodyfont=\small]{mystyle}
\theoremstyle{mystyle}
\newtheorem{dfn}{Definition}
\newtheorem{thm}{Theorem}
\newtheorem{cor}{Corollary}
\newtheorem{lem}{Lemma}
\newtheorem{rmk}{Remark}
\newtheorem{pro}{Proposition}
\usepackage{calc}

\title{Programming Project 3 Theory (Part 2)\\
\textit{The Roothaan-Hall equations and the Pople-Nesbet equations}}
\date{}
\author{}

\begin{document}
\maketitle
\vspace{-2cm}

\noindent
\subsection*{Spin and spin-orbitals}

A spin-orbital $\y$ can be decomposed as $\y(\bo{r},s)=\f(\bo{r})\w(s)$, where $\f(\bo{r})$ is a spatial orbital and $\w(s)$ is a spin function.
Spin functions live in a two-dimensional space spanned by $\{\a(s),\b(s)\}$, which are orthonormal eigenfunctions of one component of $\op{\bo{S}}$, the spin angular momentum operator.
\begin{align}
&&
  \op{S}_z\a
=
  +\fr{1}{2}\a
&&
  \op{S}_z\b
=
  -\fr{1}{2}\b
&&
  \ip{\a|\a}=\ip{\b|\b}=1
&&
  \ip{\a|\b}=\ip{\b|\a}=0
\end{align}
The ``spin-coordinate'' $s$ is either $+$ or $-$, identifying the component of $\w(s)$ along $\a(s)$ and $\b(s)$.
You can think of it as the index of a coordinate vector in spin space
\begin{align*}
&&
  \ma{\w(+)\\\w(-)}
=
  \ma{\ip{\a|\w}\\\ip{\b|\w}}
\end{align*}
so that $\a(+)=\b(-)=1$ and $\a(-)=\b(+)=0$.
Using this spin coordinate, the inner product in spin space can be defined explicitly as $\ip{\w|\w'}=\sum_s\w^*(s)\w'(s)$.
It is typical to refer to the inner product of spin functions as a ``spin integration''.

A complete set of one-electron functions (spin-orbitals) comes in $\a,\b$-pairs.
\begin{align}
\label{spin-orb-general}
&&
	\y_{2p-1}(\bo{r},s)
=
  \f_{p_\a}(\bo{r})\a(s)
&&
  \y_{2p}(\bo{r},s)
=
  \f_{p_\b}(\bo{r})\b(s)
\end{align}
The spatial components of these functions can be expanded in terms of a set of AO basis functions $\{\x_\nu\}$ as 
\begin{align}
\label{space-orb}
&&
  \f_{p_\a}
=
  \sum_\mu \x_\mu C_{\mu p_\a}
&&
  \f_{p_\b}
=
  \sum_\mu \x_\mu C_{\mu p_\b}
\end{align}
which are atom-centered Gaussian functions (cc-pVDZ, STO-3G, etc.).


\subsection*{Spin-integration of the canonical Hartree-Fock equations}

For a system with $n_\a$ spin-up and $n_\b$ spin-down electrons, the spin-orbital canonical Hartree-Fock equation takes the form
\begin{align}
\label{canonical-hf}
&
  \op{f}\y_p
=
  \ev_p\y_p
&&
  \op{f}
=
  \op{h}
+
  \sum_{i_\a}^{n_\a}
  (\op{J}_{i_\a}-\op{K}_{i_\a})
+
  \sum_{i_\b}^{n_\b}
  (\op{J}_{i_\b}-\op{K}_{i_\b})\ .
\end{align}
This equation can be expanded in the spin basis as
\begin{align*}
  \ma{
    \ip{\a|\op{f}|\a}&\ip{\a|\op{f}|\b}\\
    \ip{\b|\op{f}|\a}&\ip{\b|\op{f}|\b}}
  \ma{
    \ip{\a|\y_p}\\
    \ip{\b|\y_p}}
=
  \ev_p
  \ma{
    \ip{\a|\y_p}\\
    \ip{\b|\y_p}}
\end{align*}
where we are integrating only over the spin coordinate.
The operators in $\op{f}$ then become
\begin{align*}
&
  \op{h}
\mapsto
  \ma{
    \ip{\a|\op{h}|\a}&0\\
    0&\ip{\b|\op{h}|\b}}
&&
  \op{J}_{i_\a}
\mapsto
  \ma{
    \ip{\a|\op{J}_{i_\a}|\a}&0\\
    0&\ip{\b|\op{J}_{i_\a}|\b}}
&&
  \op{K}_{i_\a}
\mapsto
  \ma{
    \ip{\a|\op{K}_{i_\a}|\a}&0\\
    0&\makebox[\widthof{\ip{\b|\op{K}_{i_\b}|\b}}]{0}}
\\
&
&&
  \op{J}_{i_\b}
\mapsto
  \ma{
    \ip{\a|\op{J}_{i_\b}|\a}&0\\
    0&\ip{\b|\op{J}_{i_\b}|\b}}
&&
  \op{K}_{i_\b}
\mapsto
  \ma{
    \makebox[\widthof{\ip{\a|\op{K}_{i_\a}|\a}}]{0}&0\\
    0&\ip{\b|\op{K}_{i_\b}|\b}}
\end{align*}
where the core and Coulomb operators can be evaluated simply using $\ip{\w|\op{h}|\w'}=\op{h}\ip{\w|\w'}$, since these operators do not act on spin coordinates.
The exchange operator, however, does act on spin coordinates by its coordinate-swapping operation.
Hence, for example,
\begin{align*}
  \op{K}_{i_\b}(\bo{r})\a(s)
=
  \ip{\f_{i_\b}(\bo{r}')|\op{g}(\bo{r},\bo{r}')|\cdot(\bo{r}')}
  \ip{\b|\a}\
  \f_{i_\b}(\bo{r})\b(s)
=
  0
\end{align*}
where the $\cdot$ represents ``fill in spatial function here''.
This shows why $\ip{\a|\op{K}_{i_\b}|\a}=\ip{\b|\op{K}_{i_\b}|\a}=0$.
The remaining components can be derived in the same way.

Since all of the off-diagonal blocks in the spin basis vanish, we can separate equation \ref{canonical-hf} into two spin-integrated equations.
\begin{align}
\label{canonical-uhf-equation-alpha}
&
  \op{f}_\a \f_{p_\a}
=
  \ev_{p_\a} \f_{p_\a}
&&
  \op{f}_\a
=
  \op{h}
+
  \sum_{i_\a}^{n_\a}
  (\op{J}_{i_\a} - \op{K}_{i_\a})
+
  \sum_{i_\b}^{n_\b}
  \op{J}_{i_\b}
\\
\label{canonical-uhf-equation-beta}
&
  \op{f}_\b \f_{p_\b}
=
  \ev_{p_\b} \f_{p_\b}
&&
  \op{f}_\b
=
  \op{h}
+
  \sum_{i_\a}^{n_\a}
  \op{J}_{i_\a}
+
  \sum_{i_\b}^{n_\b}
  (\op{J}_{i_\b} - \op{K}_{i_\b})
\end{align}


\subsection*{RHF: The Roothaan-Hall Equations}

Assuming a closed-shell system with $n_\a=n_\b=n/2$, we can impose the restriction that $\f_{i_\a}=\f_{i_\b}$ for each pair of electrons.
Then equations \ref{canonical-uhf-equation-alpha} and \ref{canonical-uhf-equation-beta} collapse into a single expression.
\begin{align*}
  \op{f}_\textsc{r}\f_p
=
  \ev_p\f_p
&&
  \op{f}_\textsc{r}
=
  \op{h}
+
  \sum_i^{n/2}
  (2\op{J}_i - \op{K}_i)
\end{align*}
where the R stands for ``restricted''.
Expanding $\f_p$ in the AO basis and projecting by $\x_\mu$, we get the Roothaan-Hall equations
\begin{align*}
  \sum_\nu
  \ip{\x_\mu|\op{f}_\textsc{r}|\x_\nu}
  C_{\nu p}
=
  \sum_\nu
  \ip{\x_\mu|\x_\nu}C_{\mu p}\ev_p
\end{align*}
which can be written in matrix notation as
\begin{align}
\label{roothaan-hall}
&
  \bo{F}\bo{C}
=
  \bo{S}\bo{C}\bm{\ev}
&&
  F_{\mu\nu}
=
  \ip{\x_\mu|\op{f}_\textsc{r}|\x_\nu}
&&
  S_{\mu\nu}
=
  \ip{\x_\mu|\x_\nu}
&&
  (\bm{\ev})_{pq}
=
  \ev_p\d_{pq}\ .
\end{align}
Note that if the AO basis contains $m$ functions, then $\bo{C}=[C_{\mu p}]$ is an $m\times m$ matrix with each column vector containing the expansion coefficients for an orbital $\f_p$.
Also, note that only the $n/2$ MOs of lowest energy ($\ev_p$) will be ``occupied'' -- the remaining virtual orbitals will not enter into the Coulomb and exchange parts of $\op{f}_\textsc{R}$.
Expanding $\op{f}_\textsc{R}$ in its core, Coulomb, and exchange parts, we find
\begin{align*}
  F_{\mu\nu}
=&\
  \ip{\x_\mu|\op{h}|\x_\nu}
+
  \sum_i^{n/2}
  (2\ip{\x_\mu\f_i|\x_\nu\f_i}-\ip{\x_\mu\f_i|\f_i\x_\nu})
\\
=&\
  \ip{\x_\mu|\op{h}|\x_\nu}
+
  \sum_i^{n/2}
  \sum_{\rho\si}^m
  C_{\rho i}^*C_{\si i}
  (2\ip{\x_\mu\x_\rho|\x_\nu\x_\si}-\ip{\x_\mu\x_\rho|\x_\si\x_\nu})
\end{align*}
which is conveniently given in terms of a ``density matrix'' $D_{\mu\nu}$.
\begin{align}
\label{rhf-ao-basis-fock}
  F_{\mu\nu}
=&\
  \ip{\x_\mu|\op{h}|\x_\nu}
+
  \sum_{\rho\si}^m
  D_{\rho\si}
  (2\ip{\x_\mu\x_\rho|\x_\nu\x_\si}-\ip{\x_\mu\x_\rho|\x_\si\x_\nu})
&&
  D_{\mu\nu}
=
  \sum_i^{n/2}
  C_{\mu i}^*C_{\nu i}
\end{align}


\subsubsection*{Solving the Roothaan-Hall Equations}

Equation \ref{roothaan-hall} would look like an ordinary eigenvalue problem if $\bo{S}$ were an identity matrix.
This would be the case if the AO basis were orthogonal.
We can get around this problem by an algebraic trick:
if we multiply both sides of equation \ref{roothaan-hall} by $\bo{S}^{-\frac{1}{2}}$ and insert $\bo{I}=\bo{S}^{-\frac{1}{2}}\bo{S}^{\frac{1}{2}}$ between $\bo{F}$ and $\bo{C}$, we get
\begin{align}
\label{orthogonalized-roothaan-hall}
&
  \tl{\bo{F}}
  \tl{\bo{C}}
=
  \tl{\bo{C}}\bm\ev
&&
  \tl{\bo{F}}
=
  \bo{S}^{-\frac{1}{2}}\bo{F}\bo{S}^{-\frac{1}{2}}
&&
  \tl{\bo{C}}
=
  \bo{S}^{\frac{1}{2}}\bo{C}\ .
\end{align}
This is equivalent to a transformation to an orthogonalized AO basis $\tl\x_\mu=\sum_\nu\x_\nu (\bo{S}^{-\frac{1}{2}})_{\nu\mu}$ which satisfies $\ip{\tl\x_\mu|\tl\x_\nu}=\d_{\mu\nu}$.
After diagonalizing $\tl{\bo{F}}$, the ordinary MO coefficient matrix can be recovered as $\bo{C}=\bo{S}^{-\frac{1}{2}}\tl{\bo{C}}$.

Recall, however, that equation \ref{orthogonalized-roothaan-hall} is still not an ordinary eigenvalue problem because $\bo{F}$ depends on the MOs via the density matrix $\bo{D}$.
The standard procedure for solving the Roothan-Hall equations is as follows:
\begin{enumerate}
  \item Get integrals $\ip{\x_\mu|\x_\nu}, \ip{\x_\mu|\op{h}|\x_\nu}, \ip{\x_\mu\x_\nu|\x_\rho\x_\si}$ and form orthogonalizer $\bo{S}^{-\frac{1}{2}}$
  \item Guess $\bo{D}=\bo{0}$
  \item\label{loop} Build $\bo{F}$ (equation \ref{rhf-ao-basis-fock})
  \item Diagonalize $\tl{\bo{F}}=\bo{S}^{-\frac{1}{2}}\bo{F}\bo{S}^{-\frac{1}{2}}$, giving $\tl{\bo{C}}$ and $\bm\ev$
  \item Backtransform $\bo{C}=\bo{S}^{-\frac{1}{2}}\tl{\bo{C}}$
  \item Form new density matrix $D_{\mu\nu}=\sum_i^{n/2} C_{\mu i}^*C_{\nu i}$
  \item If the new $\bo{D}$ differs from the old $\bo{D}$ by more than a threshold, return to step \ref{loop}
\end{enumerate}


\subsection*{UHF: The Pople-Nesbet Equations}

For open-shell systems of arbitrary $n_\a, n_\b$, we can solve equations \ref{canonical-uhf-equation-alpha} and \ref{canonical-uhf-equation-beta} without any spin restriction.
By exactly the same procedure as was used for the Roothaan-Hall equations, we arrive at the Pople-Nesbet equations
\begin{align*}
&
  \bo{F}^\a\bo{C}^\a
=
  \bo{S}\bo{C}^\a\bm\ev^\a
&&
  F_{\mu\nu}^\a
=
  \ip{\x_\mu|\op{f}_\a|\x_\nu}
&&
  S_{\mu\nu}
=
  \ip{\x_\mu|\x_\nu}
&&
  (\bm\ev^\a)_{p_\a q_\a}
=
  \ev_{p_\a}\d_{p_\a q_\a}
\\
&
  \bo{F}^\b\bo{C}^\b
=
  \bo{S}\bo{C}^\b\bm\ev^\b
&&
  F_{\mu\nu}^\b
=
  \ip{\x_\mu|\op{f}_\b|\x_\nu}
&&
  S_{\mu\nu}
=
  \ip{\x_\mu|\x_\nu}
&&
  (\bm\ev^\b)_{p_\b q_\b}
=
  \ev_{p_\b}\d_{p_\b q_\b}
\end{align*}
where $\bo{C}^\a=[C_{\mu p_\a}]$ is an $m\times m$ matrix of MO coefficients for $\{\f_{p_\a}\}$ and $\bo{C}^\b=[C_{\mu p_\b}]$ is an $m\times m$ matrix of MO coefficients for $\{\f_{p_\b}\}$.
Expanding the $\a$ and $\b$ Fock matrices as in equation \ref{rhf-ao-basis-fock}, we find
\begin{align*}
&
  F_{\mu\nu}^\a
=
  \ip{\x_\mu|\op{h}|\x_\nu}
+
  \sum_{\rho\si}
  D_{\rho\si}^\a
  \ip{\x_\mu\x_\rho||\x_\nu\x_\si}
+
  \sum_{\rho\si}
  D_{\rho\si}^\b
  \ip{\x_\mu\x_\rho|\x_\nu\x_\si}
&&
  D_{\mu\nu}^\a
=
  \sum_{i_\a}^{n_\a} C_{\mu i_\a}^*C_{\nu i_\a}
\\
&
  F_{\mu\nu}^\b
=
  \ip{\x_\mu|\op{h}|\x_\nu}
+
  \sum_{\rho\si}
  D_{\rho\si}^\a
  \ip{\x_\mu\x_\rho|\x_\nu\x_\si}
+
  \sum_{\rho\si}
  D_{\rho\si}^\b
  \ip{\x_\mu\x_\rho||\x_\nu\x_\si}
&&
  D_{\mu\nu}^\b
=
  \sum_{i_\b}^{n_\b} C_{\mu i_\b}^*C_{\nu i_\b}
\end{align*}
where $\bo{D}^\a$ and $\bo{D}^\b$ are the $\a$ and $\b$ density matrices.
The procedure for solving the Pople-Nesbet equations is identical to the one given for RHF.
However, note that one must solve the $\a$ and $\b$ equations simultaneously because each Fock operator depends on both $\bo{C}^\a$ and $\bo{C}^\b$ (via $\bo{D}^\a$ and $\bo{D}^\b$).

\end{document}