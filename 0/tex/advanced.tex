\documentclass[fleqn]{article}
\usepackage[cm]{fullpage}
\usepackage{mathtools} %includes amsmath
\usepackage{amsfonts}
\usepackage{bm}
\usepackage{url}

\usepackage{listings}
\lstset{basicstyle=\ttfamily\small}
\lstset{literate={~} {$\sim$}{1}}
\lstset{showstringspaces=false}
\lstset{language=Python}
\newcommand{\linp}[1]{\lstinputlisting{#1}}
\newcommand{\linl}[1]{\lstinline{#1}{}}
\newcommand{\ul}[1]{\underline{#1}}

\title{Programming Project 0 (Advanced)}
\author{}
\date{}

\begin{document}

\maketitle
\vspace{-1cm}
\noindent
Implement the following Python ``magic methods'' in your \linl{Molecule} class:

\begin{center}
\begin{tabular}{p{0.1\textwidth}@{}p{0.1\textwidth}@{}p{0.1\textwidth}@{}p{0.2\textwidth}@{}p{0.5\textwidth}}
  \ul{name}      & \ul{returns} & \ul{args} & \ul{called by\footnotemark[1]} & \ul{description}
  \\
  \linl{__len__}
  & \linl{int}
  &
  & \linl{len(mol)}
  & return the ``length'' (number of atoms) of a molecule instance
  \\
  \linl{__str__}
  & \linl{str}
  &
  & \linl{print(mol)}, \linl{str(mol)}
  & represent the contents of a \linl{Molecule} object as a string in \linl{.xyz} format
  \\
  \linl{__iter__}
  &
  iterator
  &
  & \linl{for _ in mol}
  & iterates over \linl{(str, numpy.array)} \linl{tuple}s, each of which contains the atomic symbol and Cartesian coordinates of an atom in the molecule
  \\
  \linl{__sum__}
  & \linl{Molecule}
  & \linl{Molecule}
  & \linl{mol1 + mol2}
  & returns a new molecule object containing \\
\end{tabular}
\end{center}

\footnotetext[1]{Assume \linl{mol}, \linl{mol1}, and \linl{mol2} are all instances of your \linl{Molecule} class.}


\end{document}