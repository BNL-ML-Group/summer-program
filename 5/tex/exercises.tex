\documentclass[fleqn,12pt]{article}
\usepackage{mystyle}

\title{Programming Project 5 Exercises}
\author{}
\date{}

\begin{document}

\maketitle

\begin{enumerate}
  \item Derive the Hartree-Fock energy expression
\begin{align}
  \ip{\F|\op{H}_e|\F}
=
  \sum_i^n
  h_{ii}
+
  \sum_{i<j}^n
  \ip{ij||ij}
\end{align}
by expanding the Hartree-Fock determinant as
\begin{align}
&
  \F(1\cd n)
=
  \fr{1}{\sqrt{n!}}
  \sum_{i_1\cd i_n}^n
  \e_{i_1\cd i_n}
  \y_{i_1}(1)\cd\y_{i_n}(n)
\end{align}
where $\e_{i_1\cd i_n}$ is the Levi-Civita permutation tensor, given by
\begin{align}
  \e_{i_1\cd i_n}
=
  \left\{\ar{
    +1&\text{if $(i_1\cd i_n)$ is an even permutation of $(1\cd n)$}\\
    -1&\text{if $(i_1\cd i_n)$ is an odd permutation of $(1\cd n)$}\\
    0&\text{otherwise.}
  }\right.
\end{align}

  \item Let $\ol{\bo{S}}$, $\ol{\bo{T}}$, and $\ol{\bo{V}}$ be one-electron integrals with respect to the spatial AO basis $\{\x_\mu\}$.
\begin{align}
\label{space-ao-ints}
&
  \ol{S}_{\mu\nu}
=
  \ip{\x_{\mu}|\x_{\nu}}
&&
  \ol{T}_{\mu\nu}
=
  -\tfrac{1}{2}\ip{\x_{\mu}|\nabla_1^2|\x_{\nu}}
&&
  \ol{V}_{\mu\nu}
=
  \sum_A\ip{\x_{\mu}|\fr{Z_A}{|\bo{r}_1-\bo{R}_A|}|\x_{\nu}}
\end{align}
  Explain why their spin-orbital counterparts $\bo{S}$, $\bo{T}$, $\bo{V}$ have the following block-diagonal structure
\begin{align}
\label{spin-ao-ints-1}
&
  \bo{S}
=
  \ma{\ol{\bo{S}}&0\\0&\ol{\bo{S}}}
&&
  \bo{T}
=
  \ma{\ol{\bo{T}}&0\\0&\ol{\bo{T}}}
&&
  \bo{V}
=
  \ma{\ol{\bo{V}}&0\\0&\ol{\bo{V}}}
\end{align}
  in the spin-AO basis $\{\x_\mu\a\}\cup\{\x_\mu\b\}$.
  \item The two-electron integrals $(\mu\nu|\rho\si)$ make a 4-index tensor, which can be viewed as a matrix $\ol{\bo{G}}=[\ol{\bo{G}}_{\mu\nu}]$ of matrices $\ol{\bo{G}}_{\mu\nu}=[(\mu\nu|\rho\si)]$.
  Explain why the spatial AO basis integral tensor $\ol{\bo{G}}$ is related to its spin-AO counterpart $\bo{G}$ via
\begin{align}
&&
  \bo{G}
=
  \ma{\ol{\bo{G}}&0\\0&\ol{\bo{G}}}
&&
  (\bo{G})_{\mu\nu}
=
  \ma{(\ol{\bo{G}})_{\mu\nu}&0\\0&(\ol{\bo{G}})_{\mu\nu}}
\end{align}
  i.e. it has a block-diagonal structure with respect to each pair of indices.
  Note that this is one case where the index-ordering of chemist's notation has some advantages over physicist's notation.
  \item The spin-orbital Fock operator is given by
\begin{align}
  \op{f}
=
  \op{h}
+
  \sum_{i=1}^n
  (\op{J}_i - \op{K}_i)
\end{align}
  where $\op{h}$ is the one-electron (``core'') Hamiltonian and $\op{J}_i$ and $\op{K}_i$ are Coulomb and exchange operators.
  Show that its matrix elements with respect to the spin-orbital basis $\{\y_p\}$ are given by
\begin{align}
  f_{pq}
=
  h_{pq}
+
  \sum_{i=1}^n
  \ip{pi||qi}\ .
\end{align}
  \item 
  Show that, if $C_{\mu p}$ are the expansion coefficients of $\y_p$ in the spin-AO basis $\{\xi_\mu\}=\{\x_\mu\a\}\cup\{\x_\mu\b\}$
\begin{align}
  \y_p
=
  \sum_\mu \xi_\mu C_{\mu p}
\end{align}
  then the spin-AO basis Fock matrix elements are given by
\begin{align}
&
  f_{\mu\nu}
=
  h_{\mu\nu}
+
  \sum_{\rho\si}
  \ip{\mu\rho||\nu\si}D_{\rho\si}
&&
  D_{\mu\nu}
=
  \sum_{i=1}^n
  C_{\mu i}^*C_{\nu i}
\end{align}
  where $D_{\mu\nu}$ is called the Hartree-Fock density matrix.\footnote{This is the spin-AO-basis representation of the one-particle reduced density matrix of a Hartree-Fock determinant.}
\item Show that the Hartree-Fock energy can be expanded in terms of spin-AO basis integrals as follows.
\begin{align}
  \ip{\F|\op{H}_e|\F}
=
  \sum_{\mu\nu}
  h_{\mu\nu}D_{\mu\nu}
+
  \fr{1}{2}
  \sum_{\mu\nu\rho\si}
  \ip{\mu\rho||\nu\si}
  D_{\mu\nu}D_{\rho\si}
\end{align}
\end{enumerate}

\end{document}
