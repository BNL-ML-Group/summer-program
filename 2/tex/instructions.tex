\documentclass[fleqn]{article}

\usepackage{listings}
\lstset{basicstyle=\ttfamily\small}
\lstset{literate={~} {$\sim$}{1}}
\lstset{showstringspaces=false}
\lstset{language=Python}
\usepackage{scrextend}
\newcommand{\linp}[1]{\lstinputlisting{#1}{}}
\newcommand{\linl}[1]{\lstinline{#1}{}}

\usepackage[cm]{fullpage}
\usepackage{mathtools} %includes amsmath
\usepackage{amsfonts}
\usepackage{bm}
\usepackage{url}
%greek letters
\renewcommand{\a}{\alpha}    %alpha
\renewcommand{\b}{\beta}     %beta
\newcommand{\g}{\gamma}      %gamma
\newcommand{\G}{\Gamma}      %Gamma
\renewcommand{\d}{\delta}    %delta
\newcommand{\D}{\Delta}      %Delta
\newcommand{\e}{\varepsilon} %epsilon
\newcommand{\ev}{\epsilon}   %epsilon*
\newcommand{\z}{\zeta}       %zeta
\newcommand{\h}{\eta}        %eta
\renewcommand{\th}{\theta}   %theta
\newcommand{\Th}{\Theta}     %Theta
\newcommand{\io}{\iota}      %iota
\renewcommand{\k}{\kappa}    %kappa
\newcommand{\la}{\lambda}    %lambda
\newcommand{\La}{\Lambda}    %Lambda
\newcommand{\m}{\mu}         %mu
\newcommand{\n}{\nu}         %nu %xi %Xi %pi %Pi
\newcommand{\p}{\rho}        %rho
\newcommand{\si}{\sigma}     %sigma
\newcommand{\siv}{\varsigma} %sigma*
\newcommand{\Si}{\Sigma}     %Sigma
\renewcommand{\t}{\tau}      %tau
\newcommand{\up}{\upsilon}   %upsilon
\newcommand{\f}{\phi}        %phi
\newcommand{\F}{\Phi}        %Phi
\newcommand{\x}{\chi}        %chi
\newcommand{\y}{\psi}        %psi
\newcommand{\Y}{\Psi}        %Psi
\newcommand{\w}{\omega}      %omega
\newcommand{\W}{\Omega}      %Omega
%ornaments
\newcommand{\eth}{\ensuremath{^\text{th}}}
\newcommand{\rst}{\ensuremath{^\text{st}}}
\newcommand{\ond}{\ensuremath{^\text{nd}}}
\newcommand{\ord}[1]{\ensuremath{^{(#1)}}}
\newcommand{\dg}{\ensuremath{^\dagger}}
\newcommand{\bigo}{\ensuremath{\mathcal{O}}}
\newcommand{\tl}{\ensuremath{\tilde}}
\newcommand{\ol}[1]{\ensuremath{\overline{#1}}}
\newcommand{\ul}[1]{\underline{#1}}
\newcommand{\op}[1]{\ensuremath{\hat{#1}}}
\newcommand{\ot}{\ensuremath{\otimes}}
\newcommand{\wg}{\ensuremath{\wedge}}
%text
\newcommand{\tr}{\ensuremath{\hspace{1pt}\mathrm{tr}\hspace{1pt}}}
\newcommand{\Alt}{\ensuremath{\mathrm{Alt}}}
\newcommand{\sgn}{\ensuremath{\mathrm{sgn}}}
\newcommand{\occ}{\ensuremath{\mathrm{occ}}}
\newcommand{\vir}{\ensuremath{\mathrm{vir}}}
\newcommand{\spn}{\ensuremath{\mathrm{span}}}
\newcommand{\vac}{\ensuremath{\mathrm{vac}}}
\newcommand{\bs}{\ensuremath{\text{\textbackslash}}}
\newcommand{\im}{\ensuremath{\mathrm{im}\hspace{1pt}}}
\renewcommand{\sp}{\hspace{30pt}}
%dots
\newcommand{\ld}{\ensuremath{\ldots}}
\newcommand{\cd}{\ensuremath{\cdots}}
\newcommand{\vd}{\ensuremath{\vdots}}
\newcommand{\dd}{\ensuremath{\ddots}}
\newcommand{\etc}{\ensuremath{\mathinner{\mkern-1mu\cdotp\mkern-2mu\cdotp\mkern-2mu\cdotp\mkern-1mu}}}
%fonts
\newcommand{\bmit}[1]{{\bfseries\itshape\mathversion{bold}#1}}
\newcommand{\mc}[1]{\ensuremath{\mathcal{#1}}}
\newcommand{\mb}[1]{\ensuremath{\mathbb{#1}}}
\newcommand{\mf}[1]{\ensuremath{\mathfrak{#1}}}
\newcommand{\mr}[1]{\ensuremath{\mathrm{#1}}}
\newcommand{\bo}[1]{\ensuremath{\mathbf{#1}}}
%styles
\newcommand{\ts}{\textstyle}
\newcommand{\ds}{\displaystyle}
\newcommand{\phsub}{\ensuremath{_{\phantom{p}}}}
\newcommand{\phsup}{\ensuremath{^{\phantom{p}}}}
%fractions, derivatives, parentheses, brackets, etc.
\newcommand{\pr}[1]{\ensuremath{\left(#1\right)}}
\newcommand{\brk}[1]{\ensuremath{\left[#1\right]}}
\newcommand{\fr}[2]{\ensuremath{\dfrac{#1}{#2}}}
\newcommand{\pd}[2]{\ensuremath{\frac{\partial#1}{\partial#2}}}
\newcommand{\fd}[2]{\ensuremath{\frac{d #1}{d #2}}}
\newcommand{\pt}{\ensuremath{\partial}}
\newcommand{\br}[1]{\ensuremath{\langle#1|}}
\newcommand{\kt}[1]{\ensuremath{|#1\rangle}}
\newcommand{\ip}[1]{\ensuremath{\langle#1\rangle}}
%structures
\newcommand{\eqn}[1]{(\ref{#1})}
\newcommand{\ma}[1]{\ensuremath{\begin{bmatrix}#1\end{bmatrix}}}
\newcommand{\ar}[1]{\ensuremath{\begin{matrix}#1\end{matrix}}}
\newcommand{\miniar}[1]{\ensuremath{\begin{smallmatrix}#1\end{smallmatrix}}}
\newcommand{\rad}{\ensuremath{\mr{rad}}}


\title{Programming Project 2: Hessian\\
\textit{Computing the Hessian matrix by finite differences}}
\author{}
\date{}

\begin{document}

\maketitle
\vspace{-1cm}

\noindent
Compute the Hessian matrix of a molecule (equation \ref{hessian-dfn}) from single-point energies at displaced geometries.
The default displacement size should be 0.005~$a_0$.
You should import and use your \linl{Molecule} object from Project 0.
Werever it makes sense, you may want to add methods to your \linl{Molecule} class in order to clean up your code.

\subsection*{Extra Files}
\begin{tabular}{p{0.25\textwidth}@{}p{0.75\textwidth}}
  \ul{file name} & \ul{description} \\
  \linl{project2_input.dat}
  & sample RHF/cc-pVDZ input file for H$_2$O\\
  \linl{template.dat}
  & an template for generating input files; after reading this to a \linl{str}, you can fill in the geometry block using \linl{.format()}\\ 
\end{tabular}


\subsection*{Equations}
Let $N$ be the number of atoms and let $(x_A, y_A, z_A)$ be the Cartesian coordinates of the $A$\eth\ atom.
\begin{align}
\label{hessian-dfn}
  (\bo{H})_{AB}
=&\
  \pd{^2E}{X_A\pt X_B}
&
  (X_{3A-2}, X_{3A-1}, X_{3A-0})=&\ (x_A, y_A, z_A)
&&
  \text{for $A\in\{1,\ld,N\}$}
\end{align}
{\small
\begin{align}
\label{diagonal-elements}
  \pd{^2E}{X_A^2}
=&\
  \fr{E(X_A+h)+E(X_A-h)-2E(X_A)}{h^2}
&\text{for $X_A= X_B$}
\\
\nonumber
  \pd{^2E}{X_A\pt X_B}
=&\
  \fr{1}{2h^2}
  \left(E(X_A+h,X_B+h) + E(X_A-h,X_B-h) - E(X_A+h,X_B)-E(X_A-h,X_B)\right.
\\&\
\label{off-diagonal-elements}
  \hphantom{\fr{1}{2h^2}(}
  \left.- E(X_A,X_B+h) - E(X_A,X_B-h) + 2E(X_A,X_B)\right)
&\text{for $X_A\neq X_B$}
\end{align}}


\subsection*{Procedure}

\begin{enumerate}
  \item build molecule object (\linl{mol}) from \linl{molecule.xyz}
  \item build input file template (\linl{template}) from \linl{template.dat}
  % function 1
  \item\label{generate-inputs}
  \linl{def generate_inputs(mol, template, disp_size = 0.005, directory = "DISPS"):}
  \begin{enumerate}
    \item make directories with input files for the reference geometry and for the $\tfrac{3N(3N+1)}{2}$ unique displacements needed to evaluate equations \ref{diagonal-elements} and \ref{off-diagonal-elements}
  \end{enumerate}
  % function 2
  \item \linl{def run_jobs(mol, command = "psi4", directory = "DISPS"):}
  \begin{enumerate}
    \item walk through the directories created in step \ref{generate-inputs} and execute \linl{command} in each one
  \end{enumerate}
  % function 3
  \item \linl{def build_hessian(mol, energy_prefix, disp_size = 0.005, directory = "DISPS"):}
  \begin{enumerate}
    \item write a helper function to grab the energy value (as a \linl{float}) immediately following \linl{energy_prefix} in an output file (for the given template, \linl{energy_prefix} should be \linl{"@DF-RHF Final Energy:"})
    \item initialize an empty \linl{numpy.array} to hold the Hessian matrix
    \item loop over elements of the Hessian (equation \ref{hessian-dfn}) and evaluate them using equations \ref{diagonal-elements} and \ref{off-diagonal-elements}
    \item save the matrix to a file and return it at the end of the function
  \end{enumerate}
  \item import your \linl{frequencies} function from Project 1 and use it to calculate frequencies and normal modes from the return value of \linl{build_hessian}
\end{enumerate}
Possible extensions of this project: 1.~make it object-oriented; 2.~generalize it to work with programs other than \textsc{Psi4}; 3.~generalize it to allow cluster job submission as well as direct execution.


\end{document}