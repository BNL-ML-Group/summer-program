\documentclass[fleqn,11pt]{article}
\usepackage{mystyle}

\title{Programming Project 6 Exercises}
\author{}
\date{}

\begin{document}

\maketitle

\begin{enumerate}
  \item
  \bmit{Rayleigh-Schr\"odinger perturbation theory.}
  Given a $0\eth$ order approximation $\op{H}\ord{0}\Y_K\ord{0}=E_K\ord{0}\Y_K\ord{0}$ to the full Schr\"odinger equation $\op{H}\Y_K=E_K\Y_K$, the exact wavefunctions and energies of well-behaved systems can be expanded in terms of their $0\eth$ order counterparts using Rayleigh-Schr\"odinger perturbation theory.
  The $n\eth$ order wavefunction and energy corrections, $\Y_K\ord{n}$ and $E_K\ord{n}$, in the perturbation expansions, $\Y_K=\Y_K\ord{0}+\sum_{n=1}^{\infty}\Y_K\ord{n}$ and $E_K=E_K\ord{0}+\sum_{n=1}^{\infty}E_K\ord{n}$, can be obtained from the RSPT equations
\begin{align}
\label{rspt-equation}
&&
  (\op{H}\ord{0}-E_K\ord{0})\Y\ord{n}_K
=
  \sum_{p=0}^{n-1}E\ord{n-p}_K\Y\ord{p}_K
-
  \op{W}\Y\ord{n-1}_K
\end{align}
where $\op{W}\equiv\op{H}-\op{H}\ord{0}$.
Show that, assuming $\ip{\Y\ord{0}|\Y\ord{n}}=0$ for $n\geq1$ (intermediate normalization), the $n\eth$ order energy correction is given by
\begin{align}
&&
  E\ord{n}_K
=
  \ip{\Y\ord{0}_K|\op{W}|\Y\ord{n-1}_K}\ .
\end{align}


\item The $\Y\ord{1}$ RSPT equation (equation~\ref{rspt-equation} with $n=1$) is given by
\begin{align}
\label{rspt-equation-1st-order}
&&
  (\op{H}\ord{0}-E_K\ord{0})\Y_K\ord{1}
=
  (E_K\ord{1}-\op{W})\Y_K\ord{0}\ .
\end{align}
Using equation \ref{rspt-equation-1st-order}, show that the expansion coefficients of $\Y_K\ord{1}$ in the basis of $0\eth$ order solutions $\{\Y_L\ord{0}\}$
\begin{align*}
&&
  \Y_K\ord{1}
=
  \sum_{L\neq K}
  \Y_L\ord{0} c_{LK}\ord{1}
\end{align*}
are given by
\begin{align}
&&
  c_{LK}\ord{1}
=
  \fr{\ip{\Y_L\ord{0}|\op{W}|\Y_K\ord{0}}}{E_K\ord{0}-E_L\ord{0}}\ .
\end{align}

\item Show that the RSPT energy up to second order is given by
\begin{align}
&&
  E_K\ord{0}
+
  E_K\ord{1}
+
  E_K\ord{2}
=
  \ip{\Y_K\ord{0}|\op{H}|\Y_K\ord{0}}
+
  \sum_{L\neq K}
  \fr{\ip{\Y_K\ord{0}|\op{W}|\Y_L\ord{0}}\ip{\Y_L\ord{0}|\op{W}|\Y_K\ord{0}}}{E_K\ord{0}-E_L\ord{0}}\ .
\end{align}


\item
  \bmit{M\o ller-Plesset perturbation theory.}
When the exact electronic Hamiltonian
\begin{align*}
&&
  \op{H}_e
=
  \sum_{i=1}^n
  \op{h}(i)
+
  \fr{1}{2}
  \sum_{ij}^n
  \op{g}(i,j)
\end{align*}
is approximated by a sum of one-particle Fock operators $\op{f}(i)$
\begin{align*}
&&
  \op{H}_e\ord{0}
=
  \sum_{i=1}^n
  \op{f}(i)
=
  \sum_{i=1}^n
  \op{h}(i)
+
  \sum_{i=1}^n
  \sum_{k=1}^n
  (\op{J}_k(i)-\op{K}_k(i))
\end{align*}
the RSPT approach is called M\o ller-Plesset perturbation theory.
Show that, given spin-orbitals satisfying 
\begin{align}
&&
  \op{f}(1)\y_p(1)
=
  \ev_p\y_p(1)\sp \text{(canonical Hartree-Fock equations)},
\end{align}
the determinants $\{\F,\F_i^a,\F_{ij}^{ab},\F_{ijk}^{abc},\ld\}$ formed from them provide $0\eth$ order solutions of the form
\begin{align}
&
  \op{H}_e\ord{0}\F
=
  \mathcal{E}\F
&&
  \op{H}_e\ord{0}\F_i^a
=
  \mathcal{E}_i^a\F_i^a
&&
  \op{H}_e\ord{0}\F_{ij}^{ab}
=
  \mathcal{E}_{ij}^{ab}\F_{ij}^{ab}
&&
  \op{H}_e\ord{0}\F_{ijk}^{abc}
=
  \mathcal{E}_{ijk}^{abc}\F_{ijk}^{abc}
&&
  \etc
\end{align}
where $\mathcal{E}_{ijk\cd}^{abc\cd}$ is the sum of the orbital energies for $\F_{ijk\cd}^{abc\cd}$.

\item Show that the M\o ller-Plesset energy is given to second order by\\[5pt]
$\ds
  E\ord{0}
+
  E\ord{1}
+
  E\ord{2}
=
  E_0
+
  \sum_{\miniar{a\\i}}
  \fr{|\ip{\F|\op{W}|\F_i^a}|^2}{\ev_i-\ev_a}
+
  \sum_{\miniar{a<b\\i<j}}
  \fr{|\ip{\F|\op{W}|\F_{ij}^{ab}}|^2}{\ev_i+\ev_j-\ev_a-\ev_b}
+
  \sum_{\miniar{a<b<c\\i<j<k}}
  \fr{|\ip{\F|\op{W}|\F_{ijk}^{abc}}|^2}{\ev_i+\ev_j+\ev_k-\ev_a-\ev_b-\ev_c}
+
  \cd
$\\[5pt]
where $E_0=\ip{\F|\op{H}_e|\F}$ is the Hartree-Fock energy.\footnote{Note that $E_0$ is \underline{not} $\mathcal{E}$, the sum of $\F$'s orbital energies}

\item Use Slater's rules and Brillouin's theorem to show that the second-order M\o ller-Plesset (MP2) energy can be simplified to the following expression for a canonical Hartree-Fock reference determinant.
\begin{align}
&&
  E\ord{0}
+
  E\ord{1}
+
  E\ord{2}
=
  E_0
+
  \fr{1}{4}
  \sum_{ijab}
  \fr{|\ip{ij||ab}|^2}{\ev_i+\ev_j-\ev_a-\ev_b}
\end{align}
\end{enumerate}


\end{document}
